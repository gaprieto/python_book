
% Default to the notebook output style

    


% Inherit from the specified cell style.




    
\documentclass[11pt]{article}

    
    
    \usepackage[T1]{fontenc}
    % Nicer default font (+ math font) than Computer Modern for most use cases
    \usepackage{mathpazo}

    % Basic figure setup, for now with no caption control since it's done
    % automatically by Pandoc (which extracts ![](path) syntax from Markdown).
    \usepackage{graphicx}
    % We will generate all images so they have a width \maxwidth. This means
    % that they will get their normal width if they fit onto the page, but
    % are scaled down if they would overflow the margins.
    \makeatletter
    \def\maxwidth{\ifdim\Gin@nat@width>\linewidth\linewidth
    \else\Gin@nat@width\fi}
    \makeatother
    \let\Oldincludegraphics\includegraphics
    % Set max figure width to be 80% of text width, for now hardcoded.
    \renewcommand{\includegraphics}[1]{\Oldincludegraphics[width=.8\maxwidth]{#1}}
    % Ensure that by default, figures have no caption (until we provide a
    % proper Figure object with a Caption API and a way to capture that
    % in the conversion process - todo).
    \usepackage{caption}
    \DeclareCaptionLabelFormat{nolabel}{}
    \captionsetup{labelformat=nolabel}

    \usepackage{adjustbox} % Used to constrain images to a maximum size 
    \usepackage{xcolor} % Allow colors to be defined
    \usepackage{enumerate} % Needed for markdown enumerations to work
    \usepackage{geometry} % Used to adjust the document margins
    \usepackage{amsmath} % Equations
    \usepackage{amssymb} % Equations
    \usepackage{textcomp} % defines textquotesingle
    % Hack from http://tex.stackexchange.com/a/47451/13684:
    \AtBeginDocument{%
        \def\PYZsq{\textquotesingle}% Upright quotes in Pygmentized code
    }
    \usepackage{upquote} % Upright quotes for verbatim code
    \usepackage{eurosym} % defines \euro
    \usepackage[mathletters]{ucs} % Extended unicode (utf-8) support
    \usepackage[utf8x]{inputenc} % Allow utf-8 characters in the tex document
    \usepackage{fancyvrb} % verbatim replacement that allows latex
    \usepackage{grffile} % extends the file name processing of package graphics 
                         % to support a larger range 
    % The hyperref package gives us a pdf with properly built
    % internal navigation ('pdf bookmarks' for the table of contents,
    % internal cross-reference links, web links for URLs, etc.)
    \usepackage{hyperref}
    \usepackage{longtable} % longtable support required by pandoc >1.10
    \usepackage{booktabs}  % table support for pandoc > 1.12.2
    \usepackage[inline]{enumitem} % IRkernel/repr support (it uses the enumerate* environment)
    \usepackage[normalem]{ulem} % ulem is needed to support strikethroughs (\sout)
                                % normalem makes italics be italics, not underlines
    

    
    
    % Colors for the hyperref package
    \definecolor{urlcolor}{rgb}{0,.145,.698}
    \definecolor{linkcolor}{rgb}{.71,0.21,0.01}
    \definecolor{citecolor}{rgb}{.12,.54,.11}

    % ANSI colors
    \definecolor{ansi-black}{HTML}{3E424D}
    \definecolor{ansi-black-intense}{HTML}{282C36}
    \definecolor{ansi-red}{HTML}{E75C58}
    \definecolor{ansi-red-intense}{HTML}{B22B31}
    \definecolor{ansi-green}{HTML}{00A250}
    \definecolor{ansi-green-intense}{HTML}{007427}
    \definecolor{ansi-yellow}{HTML}{DDB62B}
    \definecolor{ansi-yellow-intense}{HTML}{B27D12}
    \definecolor{ansi-blue}{HTML}{208FFB}
    \definecolor{ansi-blue-intense}{HTML}{0065CA}
    \definecolor{ansi-magenta}{HTML}{D160C4}
    \definecolor{ansi-magenta-intense}{HTML}{A03196}
    \definecolor{ansi-cyan}{HTML}{60C6C8}
    \definecolor{ansi-cyan-intense}{HTML}{258F8F}
    \definecolor{ansi-white}{HTML}{C5C1B4}
    \definecolor{ansi-white-intense}{HTML}{A1A6B2}

    % commands and environments needed by pandoc snippets
    % extracted from the output of `pandoc -s`
    \providecommand{\tightlist}{%
      \setlength{\itemsep}{0pt}\setlength{\parskip}{0pt}}
    \DefineVerbatimEnvironment{Highlighting}{Verbatim}{commandchars=\\\{\}}
    % Add ',fontsize=\small' for more characters per line
    \newenvironment{Shaded}{}{}
    \newcommand{\KeywordTok}[1]{\textcolor[rgb]{0.00,0.44,0.13}{\textbf{{#1}}}}
    \newcommand{\DataTypeTok}[1]{\textcolor[rgb]{0.56,0.13,0.00}{{#1}}}
    \newcommand{\DecValTok}[1]{\textcolor[rgb]{0.25,0.63,0.44}{{#1}}}
    \newcommand{\BaseNTok}[1]{\textcolor[rgb]{0.25,0.63,0.44}{{#1}}}
    \newcommand{\FloatTok}[1]{\textcolor[rgb]{0.25,0.63,0.44}{{#1}}}
    \newcommand{\CharTok}[1]{\textcolor[rgb]{0.25,0.44,0.63}{{#1}}}
    \newcommand{\StringTok}[1]{\textcolor[rgb]{0.25,0.44,0.63}{{#1}}}
    \newcommand{\CommentTok}[1]{\textcolor[rgb]{0.38,0.63,0.69}{\textit{{#1}}}}
    \newcommand{\OtherTok}[1]{\textcolor[rgb]{0.00,0.44,0.13}{{#1}}}
    \newcommand{\AlertTok}[1]{\textcolor[rgb]{1.00,0.00,0.00}{\textbf{{#1}}}}
    \newcommand{\FunctionTok}[1]{\textcolor[rgb]{0.02,0.16,0.49}{{#1}}}
    \newcommand{\RegionMarkerTok}[1]{{#1}}
    \newcommand{\ErrorTok}[1]{\textcolor[rgb]{1.00,0.00,0.00}{\textbf{{#1}}}}
    \newcommand{\NormalTok}[1]{{#1}}
    
    % Additional commands for more recent versions of Pandoc
    \newcommand{\ConstantTok}[1]{\textcolor[rgb]{0.53,0.00,0.00}{{#1}}}
    \newcommand{\SpecialCharTok}[1]{\textcolor[rgb]{0.25,0.44,0.63}{{#1}}}
    \newcommand{\VerbatimStringTok}[1]{\textcolor[rgb]{0.25,0.44,0.63}{{#1}}}
    \newcommand{\SpecialStringTok}[1]{\textcolor[rgb]{0.73,0.40,0.53}{{#1}}}
    \newcommand{\ImportTok}[1]{{#1}}
    \newcommand{\DocumentationTok}[1]{\textcolor[rgb]{0.73,0.13,0.13}{\textit{{#1}}}}
    \newcommand{\AnnotationTok}[1]{\textcolor[rgb]{0.38,0.63,0.69}{\textbf{\textit{{#1}}}}}
    \newcommand{\CommentVarTok}[1]{\textcolor[rgb]{0.38,0.63,0.69}{\textbf{\textit{{#1}}}}}
    \newcommand{\VariableTok}[1]{\textcolor[rgb]{0.10,0.09,0.49}{{#1}}}
    \newcommand{\ControlFlowTok}[1]{\textcolor[rgb]{0.00,0.44,0.13}{\textbf{{#1}}}}
    \newcommand{\OperatorTok}[1]{\textcolor[rgb]{0.40,0.40,0.40}{{#1}}}
    \newcommand{\BuiltInTok}[1]{{#1}}
    \newcommand{\ExtensionTok}[1]{{#1}}
    \newcommand{\PreprocessorTok}[1]{\textcolor[rgb]{0.74,0.48,0.00}{{#1}}}
    \newcommand{\AttributeTok}[1]{\textcolor[rgb]{0.49,0.56,0.16}{{#1}}}
    \newcommand{\InformationTok}[1]{\textcolor[rgb]{0.38,0.63,0.69}{\textbf{\textit{{#1}}}}}
    \newcommand{\WarningTok}[1]{\textcolor[rgb]{0.38,0.63,0.69}{\textbf{\textit{{#1}}}}}
    
    
    % Define a nice break command that doesn't care if a line doesn't already
    % exist.
    \def\br{\hspace*{\fill} \\* }
    % Math Jax compatability definitions
    \def\gt{>}
    \def\lt{<}
    % Document parameters
    \title{Plotting}
    
    
    

    % Pygments definitions
    
\makeatletter
\def\PY@reset{\let\PY@it=\relax \let\PY@bf=\relax%
    \let\PY@ul=\relax \let\PY@tc=\relax%
    \let\PY@bc=\relax \let\PY@ff=\relax}
\def\PY@tok#1{\csname PY@tok@#1\endcsname}
\def\PY@toks#1+{\ifx\relax#1\empty\else%
    \PY@tok{#1}\expandafter\PY@toks\fi}
\def\PY@do#1{\PY@bc{\PY@tc{\PY@ul{%
    \PY@it{\PY@bf{\PY@ff{#1}}}}}}}
\def\PY#1#2{\PY@reset\PY@toks#1+\relax+\PY@do{#2}}

\expandafter\def\csname PY@tok@w\endcsname{\def\PY@tc##1{\textcolor[rgb]{0.73,0.73,0.73}{##1}}}
\expandafter\def\csname PY@tok@c\endcsname{\let\PY@it=\textit\def\PY@tc##1{\textcolor[rgb]{0.25,0.50,0.50}{##1}}}
\expandafter\def\csname PY@tok@cp\endcsname{\def\PY@tc##1{\textcolor[rgb]{0.74,0.48,0.00}{##1}}}
\expandafter\def\csname PY@tok@k\endcsname{\let\PY@bf=\textbf\def\PY@tc##1{\textcolor[rgb]{0.00,0.50,0.00}{##1}}}
\expandafter\def\csname PY@tok@kp\endcsname{\def\PY@tc##1{\textcolor[rgb]{0.00,0.50,0.00}{##1}}}
\expandafter\def\csname PY@tok@kt\endcsname{\def\PY@tc##1{\textcolor[rgb]{0.69,0.00,0.25}{##1}}}
\expandafter\def\csname PY@tok@o\endcsname{\def\PY@tc##1{\textcolor[rgb]{0.40,0.40,0.40}{##1}}}
\expandafter\def\csname PY@tok@ow\endcsname{\let\PY@bf=\textbf\def\PY@tc##1{\textcolor[rgb]{0.67,0.13,1.00}{##1}}}
\expandafter\def\csname PY@tok@nb\endcsname{\def\PY@tc##1{\textcolor[rgb]{0.00,0.50,0.00}{##1}}}
\expandafter\def\csname PY@tok@nf\endcsname{\def\PY@tc##1{\textcolor[rgb]{0.00,0.00,1.00}{##1}}}
\expandafter\def\csname PY@tok@nc\endcsname{\let\PY@bf=\textbf\def\PY@tc##1{\textcolor[rgb]{0.00,0.00,1.00}{##1}}}
\expandafter\def\csname PY@tok@nn\endcsname{\let\PY@bf=\textbf\def\PY@tc##1{\textcolor[rgb]{0.00,0.00,1.00}{##1}}}
\expandafter\def\csname PY@tok@ne\endcsname{\let\PY@bf=\textbf\def\PY@tc##1{\textcolor[rgb]{0.82,0.25,0.23}{##1}}}
\expandafter\def\csname PY@tok@nv\endcsname{\def\PY@tc##1{\textcolor[rgb]{0.10,0.09,0.49}{##1}}}
\expandafter\def\csname PY@tok@no\endcsname{\def\PY@tc##1{\textcolor[rgb]{0.53,0.00,0.00}{##1}}}
\expandafter\def\csname PY@tok@nl\endcsname{\def\PY@tc##1{\textcolor[rgb]{0.63,0.63,0.00}{##1}}}
\expandafter\def\csname PY@tok@ni\endcsname{\let\PY@bf=\textbf\def\PY@tc##1{\textcolor[rgb]{0.60,0.60,0.60}{##1}}}
\expandafter\def\csname PY@tok@na\endcsname{\def\PY@tc##1{\textcolor[rgb]{0.49,0.56,0.16}{##1}}}
\expandafter\def\csname PY@tok@nt\endcsname{\let\PY@bf=\textbf\def\PY@tc##1{\textcolor[rgb]{0.00,0.50,0.00}{##1}}}
\expandafter\def\csname PY@tok@nd\endcsname{\def\PY@tc##1{\textcolor[rgb]{0.67,0.13,1.00}{##1}}}
\expandafter\def\csname PY@tok@s\endcsname{\def\PY@tc##1{\textcolor[rgb]{0.73,0.13,0.13}{##1}}}
\expandafter\def\csname PY@tok@sd\endcsname{\let\PY@it=\textit\def\PY@tc##1{\textcolor[rgb]{0.73,0.13,0.13}{##1}}}
\expandafter\def\csname PY@tok@si\endcsname{\let\PY@bf=\textbf\def\PY@tc##1{\textcolor[rgb]{0.73,0.40,0.53}{##1}}}
\expandafter\def\csname PY@tok@se\endcsname{\let\PY@bf=\textbf\def\PY@tc##1{\textcolor[rgb]{0.73,0.40,0.13}{##1}}}
\expandafter\def\csname PY@tok@sr\endcsname{\def\PY@tc##1{\textcolor[rgb]{0.73,0.40,0.53}{##1}}}
\expandafter\def\csname PY@tok@ss\endcsname{\def\PY@tc##1{\textcolor[rgb]{0.10,0.09,0.49}{##1}}}
\expandafter\def\csname PY@tok@sx\endcsname{\def\PY@tc##1{\textcolor[rgb]{0.00,0.50,0.00}{##1}}}
\expandafter\def\csname PY@tok@m\endcsname{\def\PY@tc##1{\textcolor[rgb]{0.40,0.40,0.40}{##1}}}
\expandafter\def\csname PY@tok@gh\endcsname{\let\PY@bf=\textbf\def\PY@tc##1{\textcolor[rgb]{0.00,0.00,0.50}{##1}}}
\expandafter\def\csname PY@tok@gu\endcsname{\let\PY@bf=\textbf\def\PY@tc##1{\textcolor[rgb]{0.50,0.00,0.50}{##1}}}
\expandafter\def\csname PY@tok@gd\endcsname{\def\PY@tc##1{\textcolor[rgb]{0.63,0.00,0.00}{##1}}}
\expandafter\def\csname PY@tok@gi\endcsname{\def\PY@tc##1{\textcolor[rgb]{0.00,0.63,0.00}{##1}}}
\expandafter\def\csname PY@tok@gr\endcsname{\def\PY@tc##1{\textcolor[rgb]{1.00,0.00,0.00}{##1}}}
\expandafter\def\csname PY@tok@ge\endcsname{\let\PY@it=\textit}
\expandafter\def\csname PY@tok@gs\endcsname{\let\PY@bf=\textbf}
\expandafter\def\csname PY@tok@gp\endcsname{\let\PY@bf=\textbf\def\PY@tc##1{\textcolor[rgb]{0.00,0.00,0.50}{##1}}}
\expandafter\def\csname PY@tok@go\endcsname{\def\PY@tc##1{\textcolor[rgb]{0.53,0.53,0.53}{##1}}}
\expandafter\def\csname PY@tok@gt\endcsname{\def\PY@tc##1{\textcolor[rgb]{0.00,0.27,0.87}{##1}}}
\expandafter\def\csname PY@tok@err\endcsname{\def\PY@bc##1{\setlength{\fboxsep}{0pt}\fcolorbox[rgb]{1.00,0.00,0.00}{1,1,1}{\strut ##1}}}
\expandafter\def\csname PY@tok@kc\endcsname{\let\PY@bf=\textbf\def\PY@tc##1{\textcolor[rgb]{0.00,0.50,0.00}{##1}}}
\expandafter\def\csname PY@tok@kd\endcsname{\let\PY@bf=\textbf\def\PY@tc##1{\textcolor[rgb]{0.00,0.50,0.00}{##1}}}
\expandafter\def\csname PY@tok@kn\endcsname{\let\PY@bf=\textbf\def\PY@tc##1{\textcolor[rgb]{0.00,0.50,0.00}{##1}}}
\expandafter\def\csname PY@tok@kr\endcsname{\let\PY@bf=\textbf\def\PY@tc##1{\textcolor[rgb]{0.00,0.50,0.00}{##1}}}
\expandafter\def\csname PY@tok@bp\endcsname{\def\PY@tc##1{\textcolor[rgb]{0.00,0.50,0.00}{##1}}}
\expandafter\def\csname PY@tok@fm\endcsname{\def\PY@tc##1{\textcolor[rgb]{0.00,0.00,1.00}{##1}}}
\expandafter\def\csname PY@tok@vc\endcsname{\def\PY@tc##1{\textcolor[rgb]{0.10,0.09,0.49}{##1}}}
\expandafter\def\csname PY@tok@vg\endcsname{\def\PY@tc##1{\textcolor[rgb]{0.10,0.09,0.49}{##1}}}
\expandafter\def\csname PY@tok@vi\endcsname{\def\PY@tc##1{\textcolor[rgb]{0.10,0.09,0.49}{##1}}}
\expandafter\def\csname PY@tok@vm\endcsname{\def\PY@tc##1{\textcolor[rgb]{0.10,0.09,0.49}{##1}}}
\expandafter\def\csname PY@tok@sa\endcsname{\def\PY@tc##1{\textcolor[rgb]{0.73,0.13,0.13}{##1}}}
\expandafter\def\csname PY@tok@sb\endcsname{\def\PY@tc##1{\textcolor[rgb]{0.73,0.13,0.13}{##1}}}
\expandafter\def\csname PY@tok@sc\endcsname{\def\PY@tc##1{\textcolor[rgb]{0.73,0.13,0.13}{##1}}}
\expandafter\def\csname PY@tok@dl\endcsname{\def\PY@tc##1{\textcolor[rgb]{0.73,0.13,0.13}{##1}}}
\expandafter\def\csname PY@tok@s2\endcsname{\def\PY@tc##1{\textcolor[rgb]{0.73,0.13,0.13}{##1}}}
\expandafter\def\csname PY@tok@sh\endcsname{\def\PY@tc##1{\textcolor[rgb]{0.73,0.13,0.13}{##1}}}
\expandafter\def\csname PY@tok@s1\endcsname{\def\PY@tc##1{\textcolor[rgb]{0.73,0.13,0.13}{##1}}}
\expandafter\def\csname PY@tok@mb\endcsname{\def\PY@tc##1{\textcolor[rgb]{0.40,0.40,0.40}{##1}}}
\expandafter\def\csname PY@tok@mf\endcsname{\def\PY@tc##1{\textcolor[rgb]{0.40,0.40,0.40}{##1}}}
\expandafter\def\csname PY@tok@mh\endcsname{\def\PY@tc##1{\textcolor[rgb]{0.40,0.40,0.40}{##1}}}
\expandafter\def\csname PY@tok@mi\endcsname{\def\PY@tc##1{\textcolor[rgb]{0.40,0.40,0.40}{##1}}}
\expandafter\def\csname PY@tok@il\endcsname{\def\PY@tc##1{\textcolor[rgb]{0.40,0.40,0.40}{##1}}}
\expandafter\def\csname PY@tok@mo\endcsname{\def\PY@tc##1{\textcolor[rgb]{0.40,0.40,0.40}{##1}}}
\expandafter\def\csname PY@tok@ch\endcsname{\let\PY@it=\textit\def\PY@tc##1{\textcolor[rgb]{0.25,0.50,0.50}{##1}}}
\expandafter\def\csname PY@tok@cm\endcsname{\let\PY@it=\textit\def\PY@tc##1{\textcolor[rgb]{0.25,0.50,0.50}{##1}}}
\expandafter\def\csname PY@tok@cpf\endcsname{\let\PY@it=\textit\def\PY@tc##1{\textcolor[rgb]{0.25,0.50,0.50}{##1}}}
\expandafter\def\csname PY@tok@c1\endcsname{\let\PY@it=\textit\def\PY@tc##1{\textcolor[rgb]{0.25,0.50,0.50}{##1}}}
\expandafter\def\csname PY@tok@cs\endcsname{\let\PY@it=\textit\def\PY@tc##1{\textcolor[rgb]{0.25,0.50,0.50}{##1}}}

\def\PYZbs{\char`\\}
\def\PYZus{\char`\_}
\def\PYZob{\char`\{}
\def\PYZcb{\char`\}}
\def\PYZca{\char`\^}
\def\PYZam{\char`\&}
\def\PYZlt{\char`\<}
\def\PYZgt{\char`\>}
\def\PYZsh{\char`\#}
\def\PYZpc{\char`\%}
\def\PYZdl{\char`\$}
\def\PYZhy{\char`\-}
\def\PYZsq{\char`\'}
\def\PYZdq{\char`\"}
\def\PYZti{\char`\~}
% for compatibility with earlier versions
\def\PYZat{@}
\def\PYZlb{[}
\def\PYZrb{]}
\makeatother


    % Exact colors from NB
    \definecolor{incolor}{rgb}{0.0, 0.0, 0.5}
    \definecolor{outcolor}{rgb}{0.545, 0.0, 0.0}



    
    % Prevent overflowing lines due to hard-to-break entities
    \sloppy 
    % Setup hyperref package
    \hypersetup{
      breaklinks=true,  % so long urls are correctly broken across lines
      colorlinks=true,
      urlcolor=urlcolor,
      linkcolor=linkcolor,
      citecolor=citecolor,
      }
    % Slightly bigger margins than the latex defaults
    
    \geometry{verbose,tmargin=1in,bmargin=1in,lmargin=1in,rmargin=1in}
    
    

    \begin{document}
    
    
    \maketitle
    
    

    
    \hypertarget{gruxe1ficas-de-datos}{%
\section{Gráficas de datos}\label{gruxe1ficas-de-datos}}

Una de las grandes ventajas de aprender programación en Python a
diferencia de Fortran o C/C++, es que Python tiene paquetes para generar
figuras de alta calidad (aunque yo recomiendo aprender igual Fortran o
C/C++). El poder de los paquetes de graficas en Python es mucho mayor a
lo que se puede presentar en este curso y es su trabajo aprender más de
lo que se muestra acá.

Lo primero, es tener claro que una figura debe tener toda la información
necesaria para que cualquier persona que la vea lo pueda enender. Por
eso, cada eje debe estar descrito, y si se tienen más de un tipo de
datos, se debe explicar que significa cada uno. Para la presentación de
resultados científicos y especialmente en geociencias, es importante
generar figuras que muestren la información de manera correcta, veraz y
con buena resolución. No es aceptable entregar figuras sin ejes
explicados, o escalas si es necesario.

Por último, aunque se muestran algunos ejemplos de figuras en 3D, éstas
en muchos casos (aunque bonitas) no son útiles para presentar la
información. Las páginas de revistas o la pantalla del computador es 2D,
por lo que no es fácil ver la tercera dimensión. A veces es mejor un
mapa de contornos, que mostrar la topografía en 3D en una figura.

El paquete que se utilizará es matplotlib y específicamente pyplot. Como
referencia se puede encontrar en https://matplotlib.org. Para la
generación de mapas vamos a utilizar cartopy
https://scitools.org.uk/cartopy/docs/latest/

    \hypertarget{graficas-1d2d}{%
\section{Graficas 1D/2D}\label{graficas-1d2d}}

En el primer ejemplo, se muestra una figura con varios arreglos de datos
y la forma de graficarlos en una sola figura. En esto se puede ver la
forma de generar diferentes tipos de líneas, la leyenda y poner los
límites para uno de los ejes. Finalmente, como se puede guardar la
figura a un formato definido, que en este caso es un .png. Otros
formatos incluyen .ps, .eps, .pdf, .svg, etc.

En los ejemplos de abajo, los comandos plt.savefig y plt.show() no se
muestran para evitar ocupar demasiado espacio. Los códigos completos (en
la página del curso) si muestran todos los comandos.

    \begin{Verbatim}[commandchars=\\\{\}]
{\color{incolor}In [{\color{incolor}10}]:} \PY{c+c1}{\PYZsh{} plot\PYZus{}lines.py}
         \PY{c+c1}{\PYZsh{} Example of simple line plotting}
         
         \PY{k+kn}{import} \PY{n+nn}{numpy} \PY{k}{as} \PY{n+nn}{np}
         \PY{k+kn}{import} \PY{n+nn}{matplotlib}\PY{n+nn}{.}\PY{n+nn}{pyplot} \PY{k}{as} \PY{n+nn}{plt}
         
         \PY{n}{x}  \PY{o}{=} \PY{n}{np}\PY{o}{.}\PY{n}{linspace}\PY{p}{(}\PY{o}{\PYZhy{}}\PY{l+m+mf}{10.}\PY{p}{,} \PY{l+m+mf}{10.}\PY{p}{,} \PY{n}{num}\PY{o}{=}\PY{l+m+mi}{20}\PY{p}{)}
         \PY{n}{y1} \PY{o}{=} \PY{n}{x}\PY{o}{*}\PY{l+m+mi}{0} \PY{o}{+} \PY{l+m+mi}{1}
         \PY{n}{y2} \PY{o}{=} \PY{l+m+mf}{0.2}\PY{o}{*}\PY{n}{x} \PY{o}{+} \PY{l+m+mf}{0.1}
         \PY{n}{y3} \PY{o}{=} \PY{l+m+mf}{0.05}\PY{o}{*}\PY{n}{x}\PY{o}{*}\PY{o}{*}\PY{l+m+mi}{2} \PY{o}{+} \PY{l+m+mf}{0.1}\PY{o}{*}\PY{n}{x} \PY{o}{\PYZhy{}} \PY{l+m+mf}{0.1}
         \PY{n}{y4} \PY{o}{=} \PY{o}{\PYZhy{}}\PY{l+m+mf}{0.002}\PY{o}{*}\PY{n}{x}\PY{o}{*}\PY{o}{*}\PY{l+m+mi}{3} \PY{o}{+} \PY{l+m+mf}{0.01}\PY{o}{*}\PY{n}{x}\PY{o}{*}\PY{o}{*}\PY{l+m+mi}{2} \PY{o}{\PYZhy{}} \PY{l+m+mf}{0.05}\PY{o}{*}\PY{n}{x} \PY{o}{+} \PY{l+m+mf}{0.1}
         \PY{n}{y5} \PY{o}{=} \PY{o}{+}\PY{l+m+mf}{1e\PYZhy{}4}\PY{o}{*}\PY{n}{x}\PY{o}{*}\PY{o}{*}\PY{l+m+mi}{4} \PY{o}{+}\PY{l+m+mf}{0.002}\PY{o}{*}\PY{n}{x}\PY{o}{*}\PY{o}{*}\PY{l+m+mi}{3} \PY{o}{+} \PY{l+m+mf}{0.01}\PY{o}{*}\PY{n}{x}\PY{o}{*}\PY{o}{*}\PY{l+m+mi}{2} \PY{o}{\PYZhy{}} \PY{l+m+mf}{0.05}\PY{o}{*}\PY{n}{x} \PY{o}{+} \PY{l+m+mf}{0.1}
         
         
         \PY{n}{plt}\PY{o}{.}\PY{n}{plot}\PY{p}{(}\PY{n}{x}\PY{p}{,}\PY{n}{y1}\PY{p}{,}\PY{n}{marker}\PY{o}{=}\PY{l+s+s1}{\PYZsq{}}\PY{l+s+s1}{\PYZca{}}\PY{l+s+s1}{\PYZsq{}}\PY{p}{,}\PY{n}{linestyle}\PY{o}{=}\PY{l+s+s1}{\PYZsq{}}\PY{l+s+s1}{ }\PY{l+s+s1}{\PYZsq{}}\PY{p}{,}\PY{n}{label}\PY{o}{=}\PY{l+s+s1}{\PYZsq{}}\PY{l+s+s1}{y1}\PY{l+s+s1}{\PYZsq{}}\PY{p}{)}
         \PY{n}{plt}\PY{o}{.}\PY{n}{plot}\PY{p}{(}\PY{n}{x}\PY{p}{,}\PY{n}{y2}\PY{p}{,}\PY{l+s+s1}{\PYZsq{}}\PY{l+s+s1}{:}\PY{l+s+s1}{\PYZsq{}}\PY{p}{,}\PY{n}{label}\PY{o}{=}\PY{l+s+s1}{\PYZsq{}}\PY{l+s+s1}{y2}\PY{l+s+s1}{\PYZsq{}}\PY{p}{)}
         \PY{n}{plt}\PY{o}{.}\PY{n}{plot}\PY{p}{(}\PY{n}{x}\PY{p}{,}\PY{n}{y3}\PY{p}{,}\PY{l+s+s1}{\PYZsq{}}\PY{l+s+s1}{\PYZhy{}\PYZhy{}}\PY{l+s+s1}{\PYZsq{}}\PY{p}{,}\PY{n}{label}\PY{o}{=}\PY{l+s+s1}{\PYZsq{}}\PY{l+s+s1}{y3}\PY{l+s+s1}{\PYZsq{}}\PY{p}{)}
         \PY{n}{plt}\PY{o}{.}\PY{n}{plot}\PY{p}{(}\PY{n}{x}\PY{p}{,}\PY{n}{y4}\PY{p}{,}\PY{l+s+s1}{\PYZsq{}}\PY{l+s+s1}{\PYZhy{}.}\PY{l+s+s1}{\PYZsq{}}\PY{p}{,}\PY{n}{label}\PY{o}{=}\PY{l+s+s1}{\PYZsq{}}\PY{l+s+s1}{y4}\PY{l+s+s1}{\PYZsq{}}\PY{p}{)}
         \PY{n}{plt}\PY{o}{.}\PY{n}{plot}\PY{p}{(}\PY{n}{x}\PY{p}{,}\PY{n}{y5}\PY{p}{,}\PY{l+s+s1}{\PYZsq{}}\PY{l+s+s1}{\PYZhy{}}\PY{l+s+s1}{\PYZsq{}}\PY{p}{,}\PY{n}{label}\PY{o}{=}\PY{l+s+s1}{\PYZsq{}}\PY{l+s+s1}{y5}\PY{l+s+s1}{\PYZsq{}}\PY{p}{)}
         
         \PY{n}{plt}\PY{o}{.}\PY{n}{xlabel}\PY{p}{(}\PY{l+s+s1}{\PYZsq{}}\PY{l+s+s1}{X}\PY{l+s+s1}{\PYZsq{}}\PY{p}{)}
         \PY{n}{plt}\PY{o}{.}\PY{n}{ylabel}\PY{p}{(}\PY{l+s+s1}{\PYZsq{}}\PY{l+s+s1}{Y}\PY{l+s+s1}{\PYZsq{}}\PY{p}{)}
         
         \PY{n}{plt}\PY{o}{.}\PY{n}{title}\PY{p}{(}\PY{l+s+s1}{\PYZsq{}}\PY{l+s+s1}{X vs Y}\PY{l+s+s1}{\PYZsq{}}\PY{p}{)}
         \PY{n}{plt}\PY{o}{.}\PY{n}{ylim}\PY{p}{(}\PY{p}{(}\PY{o}{\PYZhy{}}\PY{l+m+mi}{3}\PY{p}{,} \PY{l+m+mi}{8}\PY{p}{)}\PY{p}{)}
         \PY{n}{plt}\PY{o}{.}\PY{n}{legend}\PY{p}{(}\PY{p}{)}
         \PY{n}{plt}\PY{o}{.}\PY{n}{savefig}\PY{p}{(}\PY{l+s+s1}{\PYZsq{}}\PY{l+s+s1}{fig1.png}\PY{l+s+s1}{\PYZsq{}}\PY{p}{)}
         \PY{n}{plt}\PY{o}{.}\PY{n}{show}\PY{p}{(}\PY{p}{)}
\end{Verbatim}


    \begin{center}
    \adjustimage{max size={0.9\linewidth}{0.9\paperheight}}{output_2_0.png}
    \end{center}
    { \hspace*{\fill} \\}
    
    Los formatos de las líneas y de los markers es muy variada y está por
fuera del objetivo del curso. Una explicación más precisa se puede
encontrar en la documentación de matplotlib. El espesor de la línea, el
relleno de los símbolos, etc. se puede cambiar.

Como se muestra en el ejemplo anterior, se muestra un número de comandos
para generar la figura (plt.plot) con escala lineal-lineal, aunque
también se puede usar semilogx o semilogy o loglog. Se puede adicionar
texto a cada eje, título de la figura, leyenda de los símbolos y
finalmente guardar la figura y mostrar la figura en pantalla.

    \hypertarget{subplots-y-barras-de-error}{%
\subsection{Subplots y barras de
error}\label{subplots-y-barras-de-error}}

    En el siguiente ejemplo se muestra el uso de figuras para mostrar datos
reales, con barras de error. La idea es mostrar una serie de datos (que
pueden representar datos tomados en campo) en función de la posición
(x). La Figura 7.2 muestra dos paneles (ver comando plt.subplots), uno
con los datos brutos (datos con errores aleatorios) y una curva que
representa el modelo que representa los datos. Sin embargo, en la vida
real cada dato tiene una incertidumbre que muchas veces se representa
con una barra de error (por ejemplo mostrando el \(\pm \sigma\)). Esto
se puede mostrar con el comando \texttt{plt.errorbar(x,y,yerr=yerr)}.

    \begin{Verbatim}[commandchars=\\\{\}]
{\color{incolor}In [{\color{incolor}11}]:} \PY{c+c1}{\PYZsh{} plot\PYZus{}errorbar.py}
         \PY{c+c1}{\PYZsh{} Simple code to plot some data with }
         \PY{c+c1}{\PYZsh{} error bars, and synthetic curve}
         
         \PY{k+kn}{import} \PY{n+nn}{numpy} \PY{k}{as} \PY{n+nn}{np}
         \PY{k+kn}{import} \PY{n+nn}{matplotlib}\PY{n+nn}{.}\PY{n+nn}{pyplot} \PY{k}{as} \PY{n+nn}{plt}
         
         \PY{n}{N}    \PY{o}{=} \PY{l+m+mi}{20}
         \PY{n}{x}    \PY{o}{=} \PY{n}{np}\PY{o}{.}\PY{n}{random}\PY{o}{.}\PY{n}{rand}\PY{p}{(}\PY{n}{N}\PY{p}{)}\PY{o}{*}\PY{l+m+mi}{4}
         \PY{n}{x}    \PY{o}{=} \PY{n}{np}\PY{o}{.}\PY{n}{sort}\PY{p}{(}\PY{n}{x}\PY{p}{)}
         \PY{n}{sig}  \PY{o}{=} \PY{l+m+mf}{0.01} \PY{o}{+} \PY{l+m+mf}{0.1}\PY{o}{*}\PY{n}{np}\PY{o}{.}\PY{n}{sqrt}\PY{p}{(}\PY{n}{x}\PY{p}{)}
         \PY{n}{yerr} \PY{o}{=} \PY{n}{np}\PY{o}{.}\PY{n}{random}\PY{o}{.}\PY{n}{normal}\PY{p}{(}\PY{l+m+mi}{0}\PY{p}{,}\PY{l+m+mf}{0.3}\PY{p}{,}\PY{n}{size}\PY{o}{=}\PY{n}{N}\PY{p}{)}\PY{o}{*}\PY{n}{sig} 
         \PY{n}{y}    \PY{o}{=} \PY{n}{np}\PY{o}{.}\PY{n}{exp}\PY{p}{(}\PY{o}{\PYZhy{}}\PY{n}{x}\PY{p}{)} \PY{o}{+} \PY{n}{yerr}
         
         \PY{n}{x0} \PY{o}{=} \PY{n}{np}\PY{o}{.}\PY{n}{linspace}\PY{p}{(}\PY{l+m+mf}{0.0}\PY{p}{,} \PY{l+m+mf}{4.0}\PY{p}{)}
         \PY{n}{y0}   \PY{o}{=} \PY{n}{np}\PY{o}{.}\PY{n}{exp}\PY{p}{(}\PY{o}{\PYZhy{}}\PY{n}{x0}\PY{p}{)}
         
         \PY{c+c1}{\PYZsh{} Plot the data}
         
         \PY{n}{f}\PY{p}{,} \PY{n}{axs} \PY{o}{=} \PY{n}{plt}\PY{o}{.}\PY{n}{subplots}\PY{p}{(}\PY{l+m+mi}{1}\PY{p}{,} \PY{l+m+mi}{2}\PY{p}{)}
         
         \PY{n}{ax1} \PY{o}{=} \PY{n}{axs}\PY{p}{[}\PY{l+m+mi}{0}\PY{p}{]}
         \PY{n}{ax1}\PY{o}{.}\PY{n}{scatter}\PY{p}{(}\PY{n}{x}\PY{p}{,} \PY{n}{y}\PY{p}{)}
         \PY{n}{ax1}\PY{o}{.}\PY{n}{plot}\PY{p}{(}\PY{n}{x0}\PY{p}{,}\PY{n}{y0}\PY{p}{,}\PY{n}{color}\PY{o}{=}\PY{l+s+s1}{\PYZsq{}}\PY{l+s+s1}{r}\PY{l+s+s1}{\PYZsq{}}\PY{p}{)}
         \PY{n}{ax1}\PY{o}{.}\PY{n}{set\PYZus{}xlabel}\PY{p}{(}\PY{l+s+s1}{\PYZsq{}}\PY{l+s+s1}{X}\PY{l+s+s1}{\PYZsq{}}\PY{p}{)}
         \PY{n}{ax1}\PY{o}{.}\PY{n}{set\PYZus{}ylabel}\PY{p}{(}\PY{l+s+s1}{\PYZsq{}}\PY{l+s+s1}{Y}\PY{l+s+s1}{\PYZsq{}}\PY{p}{)}
         \PY{n}{ax1}\PY{o}{.}\PY{n}{set\PYZus{}ylim}\PY{p}{(}\PY{p}{[}\PY{o}{\PYZhy{}}\PY{l+m+mf}{0.2} \PY{p}{,} \PY{l+m+mf}{1.1}\PY{p}{]}\PY{p}{)}
         
         \PY{n}{ax2} \PY{o}{=} \PY{n}{axs}\PY{p}{[}\PY{l+m+mi}{1}\PY{p}{]}
         \PY{n}{ax2}\PY{o}{.}\PY{n}{errorbar}\PY{p}{(}\PY{n}{x}\PY{p}{,} \PY{n}{y}\PY{p}{,}\PY{n}{yerr}\PY{o}{=}\PY{n}{sig}\PY{p}{,}\PY{n}{fmt}\PY{o}{=}\PY{l+s+s1}{\PYZsq{}}\PY{l+s+s1}{o}\PY{l+s+s1}{\PYZsq{}}\PY{p}{)}
         \PY{n}{ax2}\PY{o}{.}\PY{n}{plot}\PY{p}{(}\PY{n}{x0}\PY{p}{,}\PY{n}{y0}\PY{p}{,}\PY{n}{color}\PY{o}{=}\PY{l+s+s1}{\PYZsq{}}\PY{l+s+s1}{r}\PY{l+s+s1}{\PYZsq{}}\PY{p}{)}
         \PY{n}{ax2}\PY{o}{.}\PY{n}{set\PYZus{}xlabel}\PY{p}{(}\PY{l+s+s1}{\PYZsq{}}\PY{l+s+s1}{X}\PY{l+s+s1}{\PYZsq{}}\PY{p}{)}
         \PY{n}{ax2}\PY{o}{.}\PY{n}{yaxis}\PY{o}{.}\PY{n}{tick\PYZus{}right}\PY{p}{(}\PY{p}{)}
         \PY{n}{ax2}\PY{o}{.}\PY{n}{yaxis}\PY{o}{.}\PY{n}{set\PYZus{}label\PYZus{}position}\PY{p}{(}\PY{l+s+s2}{\PYZdq{}}\PY{l+s+s2}{right}\PY{l+s+s2}{\PYZdq{}}\PY{p}{)}
         \PY{n}{ax2}\PY{o}{.}\PY{n}{set\PYZus{}ylabel}\PY{p}{(}\PY{l+s+s1}{\PYZsq{}}\PY{l+s+s1}{Y}\PY{l+s+s1}{\PYZsq{}}\PY{p}{)}
         \PY{n}{ax2}\PY{o}{.}\PY{n}{set\PYZus{}ylim}\PY{p}{(}\PY{p}{[}\PY{o}{\PYZhy{}}\PY{l+m+mf}{0.2} \PY{p}{,} \PY{l+m+mf}{1.1}\PY{p}{]}\PY{p}{)}
         
         \PY{n}{plt}\PY{o}{.}\PY{n}{savefig}\PY{p}{(}\PY{l+s+s1}{\PYZsq{}}\PY{l+s+s1}{fig2.png}\PY{l+s+s1}{\PYZsq{}}\PY{p}{)}
         \PY{n}{plt}\PY{o}{.}\PY{n}{show}\PY{p}{(}\PY{p}{)} 
\end{Verbatim}


    \begin{center}
    \adjustimage{max size={0.9\linewidth}{0.9\paperheight}}{output_6_0.png}
    \end{center}
    { \hspace*{\fill} \\}
    
    El código plot\_errorbar.py muestra otra forma de trabajar con los
módulos de gráficas de Python. Note que primero se genera una figura con
subpaneles \texttt{plt.subplots}, donde se genera una variable axs que
se usa para generar las figuras de cada subpanel, usando
\texttt{ax1\ =\ axs{[}0{]}}, para el panel 1, y
\texttt{ax2\ =\ axs{[}1{]}} para el segundo panel.

Para definir el título de las figuras, los ejes, etc, se usa otra forma
de comandos para definirlos, como se ve en

\begin{verbatim}
ax2.set_xlabel(’X’)
ax2.yaxis.tick_right()
ax2.yaxis.set_label_position("right")
ax2.set_ylabel(’Y’)
ax2.set_ylim([-0.2 , 1.1])
\end{verbatim}

donde incluso se puede poner la posición del eje \texttt{Y} al lado
derecho, para hacer la figura más legible.

    \begin{Verbatim}[commandchars=\\\{\}]
{\color{incolor}In [{\color{incolor}45}]:} \PY{c+c1}{\PYZsh{}}
         \PY{c+c1}{\PYZsh{} Gráfica de una serie de Fourier for boxcar}
         \PY{c+c1}{\PYZsh{}}
         
         \PY{k+kn}{import} \PY{n+nn}{numpy} \PY{k}{as} \PY{n+nn}{np}
         \PY{k+kn}{import} \PY{n+nn}{matplotlib}\PY{n+nn}{.}\PY{n+nn}{pyplot} \PY{k}{as} \PY{n+nn}{plt}
         
         \PY{c+c1}{\PYZsh{} Create X vector}
         \PY{n}{pi} \PY{o}{=} \PY{n}{np}\PY{o}{.}\PY{n}{pi}
         \PY{n}{x} \PY{o}{=} \PY{n}{np}\PY{o}{.}\PY{n}{linspace}\PY{p}{(}\PY{o}{\PYZhy{}}\PY{n}{pi}\PY{p}{,}\PY{n}{pi}\PY{p}{,}\PY{l+m+mi}{101}\PY{p}{)}
         
         \PY{c+c1}{\PYZsh{} Exact boxcar}
         \PY{n}{f} \PY{o}{=} \PY{n}{np}\PY{o}{.}\PY{n}{ones}\PY{p}{(}\PY{n+nb}{len}\PY{p}{(}\PY{n}{x}\PY{p}{)}\PY{p}{)}
         \PY{n}{iloc} \PY{o}{=} \PY{n}{np}\PY{o}{.}\PY{n}{argwhere}\PY{p}{(}\PY{n}{x}\PY{o}{\PYZlt{}}\PY{l+m+mi}{0}\PY{p}{)}
         \PY{n}{f}\PY{p}{[}\PY{n}{iloc}\PY{p}{]} \PY{o}{=} \PY{o}{\PYZhy{}}\PY{l+m+mi}{1}
         
         \PY{c+c1}{\PYZsh{} Different odd terms of sin function}
         \PY{n}{s01} \PY{o}{=} \PY{n}{np}\PY{o}{.}\PY{n}{sin}\PY{p}{(}\PY{n}{x}\PY{p}{)}
         \PY{n}{s03} \PY{o}{=} \PY{n}{np}\PY{o}{.}\PY{n}{sin}\PY{p}{(}\PY{l+m+mi}{3}\PY{o}{*}\PY{n}{x}\PY{p}{)}\PY{o}{/}\PY{l+m+mf}{3.}
         \PY{n}{s05} \PY{o}{=} \PY{n}{np}\PY{o}{.}\PY{n}{sin}\PY{p}{(}\PY{l+m+mi}{5}\PY{o}{*}\PY{n}{x}\PY{p}{)}\PY{o}{/}\PY{l+m+mf}{5.}
         \PY{n}{s07} \PY{o}{=} \PY{n}{np}\PY{o}{.}\PY{n}{sin}\PY{p}{(}\PY{l+m+mi}{7}\PY{o}{*}\PY{n}{x}\PY{p}{)}\PY{o}{/}\PY{l+m+mf}{7.}
         \PY{n}{s09} \PY{o}{=} \PY{n}{np}\PY{o}{.}\PY{n}{sin}\PY{p}{(}\PY{l+m+mi}{9}\PY{o}{*}\PY{n}{x}\PY{p}{)}\PY{o}{/}\PY{l+m+mf}{9.}
         \PY{n}{s11} \PY{o}{=} \PY{n}{np}\PY{o}{.}\PY{n}{sin}\PY{p}{(}\PY{l+m+mi}{11}\PY{o}{*}\PY{n}{x}\PY{p}{)}\PY{o}{/}\PY{l+m+mf}{11.}
         
         \PY{c+c1}{\PYZsh{} Function of Fourier series}
         \PY{n}{y1} \PY{o}{=} \PY{p}{(}\PY{l+m+mi}{4}\PY{o}{/}\PY{n}{pi}\PY{p}{)}\PY{o}{*} \PY{p}{(}\PY{n}{s01}\PY{o}{+}\PY{n}{s03}\PY{p}{)}
         \PY{n}{y2} \PY{o}{=} \PY{p}{(}\PY{l+m+mi}{4}\PY{o}{/}\PY{n}{pi}\PY{p}{)}\PY{o}{*} \PY{p}{(}\PY{n}{s01}\PY{o}{+}\PY{n}{s03}\PY{o}{+}\PY{n}{s05}\PY{o}{+}\PY{n}{s07}\PY{p}{)}
         \PY{n}{y3} \PY{o}{=} \PY{p}{(}\PY{l+m+mi}{4}\PY{o}{/}\PY{n}{pi}\PY{p}{)}\PY{o}{*} \PY{p}{(}\PY{n}{s01}\PY{o}{+}\PY{n}{s03}\PY{o}{+}\PY{n}{s05}\PY{o}{+}\PY{n}{s07}\PY{o}{+}\PY{n}{s09}\PY{o}{+}\PY{n}{s11}\PY{p}{)}
         
         
         \PY{c+c1}{\PYZsh{} Plot}
         \PY{n}{fig}  \PY{o}{=} \PY{n}{plt}\PY{o}{.}\PY{n}{figure}\PY{p}{(}\PY{p}{)}
         \PY{n}{ax} \PY{o}{=} \PY{n}{fig}\PY{o}{.}\PY{n}{add\PYZus{}subplot}\PY{p}{(}\PY{l+m+mi}{111}\PY{p}{)}
         \PY{n}{ax}\PY{o}{.}\PY{n}{plot}\PY{p}{(}\PY{n}{x}\PY{p}{,}\PY{n}{f}\PY{p}{,}\PY{l+s+s1}{\PYZsq{}}\PY{l+s+s1}{b\PYZhy{}}\PY{l+s+s1}{\PYZsq{}}  \PY{p}{,}\PY{n}{lw}\PY{o}{=}\PY{l+m+mi}{3}\PY{p}{,}\PY{n}{label}\PY{o}{=}\PY{l+s+s1}{\PYZsq{}}\PY{l+s+s1}{Boxcar}\PY{l+s+s1}{\PYZsq{}}\PY{p}{)}
         \PY{n}{ax}\PY{o}{.}\PY{n}{plot}\PY{p}{(}\PY{n}{x}\PY{p}{,}\PY{n}{y1}\PY{p}{,}\PY{l+s+s1}{\PYZsq{}}\PY{l+s+s1}{c\PYZhy{}\PYZhy{}}\PY{l+s+s1}{\PYZsq{}}\PY{p}{,}\PY{n}{lw}\PY{o}{=}\PY{l+m+mi}{2}\PY{p}{,}\PY{n}{label}\PY{o}{=}\PY{l+s+s1}{\PYZsq{}}\PY{l+s+s1}{2 terms}\PY{l+s+s1}{\PYZsq{}}\PY{p}{)}
         \PY{n}{ax}\PY{o}{.}\PY{n}{plot}\PY{p}{(}\PY{n}{x}\PY{p}{,}\PY{n}{y2}\PY{p}{,}\PY{l+s+s1}{\PYZsq{}}\PY{l+s+s1}{r\PYZhy{}}\PY{l+s+s1}{\PYZsq{}} \PY{p}{,}\PY{n}{lw}\PY{o}{=}\PY{l+m+mi}{2}\PY{p}{,}\PY{n}{label}\PY{o}{=}\PY{l+s+s1}{\PYZsq{}}\PY{l+s+s1}{4 terms}\PY{l+s+s1}{\PYZsq{}}\PY{p}{)}
         \PY{n}{ax}\PY{o}{.}\PY{n}{plot}\PY{p}{(}\PY{n}{x}\PY{p}{,}\PY{n}{y3}\PY{p}{,}\PY{l+s+s1}{\PYZsq{}}\PY{l+s+s1}{b:}\PY{l+s+s1}{\PYZsq{}} \PY{p}{,}\PY{n}{lw}\PY{o}{=}\PY{l+m+mi}{2}\PY{p}{,}\PY{n}{label}\PY{o}{=}\PY{l+s+s1}{\PYZsq{}}\PY{l+s+s1}{6 terms}\PY{l+s+s1}{\PYZsq{}}\PY{p}{)}
         
         \PY{n}{ax}\PY{o}{.}\PY{n}{set\PYZus{}title}\PY{p}{(}\PY{l+s+sa}{r}\PY{l+s+s1}{\PYZsq{}}\PY{l+s+s1}{\PYZdl{}}\PY{l+s+s1}{\PYZbs{}}\PY{l+s+s1}{sum\PYZus{}}\PY{l+s+s1}{\PYZob{}}\PY{l+s+s1}{i=1\PYZcb{}\PYZca{}}\PY{l+s+si}{\PYZob{}N\PYZcb{}}\PY{l+s+s1}{ }\PY{l+s+s1}{\PYZbs{}}\PY{l+s+s1}{frac}\PY{l+s+s1}{\PYZob{}}\PY{l+s+s1}{sin(i*x)\PYZcb{}}\PY{l+s+si}{\PYZob{}i\PYZcb{}}\PY{l+s+s1}{\PYZdl{} for \PYZdl{}i\PYZdl{} odd}\PY{l+s+s1}{\PYZsq{}}\PY{p}{)}
         \PY{n}{ax}\PY{o}{.}\PY{n}{set\PYZus{}xlabel}\PY{p}{(}\PY{l+s+s1}{\PYZsq{}}\PY{l+s+s1}{X}\PY{l+s+s1}{\PYZsq{}}\PY{p}{)}
         \PY{n}{ax}\PY{o}{.}\PY{n}{set\PYZus{}ylabel}\PY{p}{(}\PY{l+s+s1}{\PYZsq{}}\PY{l+s+s1}{Fouriewr series sum}\PY{l+s+s1}{\PYZsq{}}\PY{p}{)}
         \PY{n}{ax}\PY{o}{.}\PY{n}{legend}\PY{p}{(}\PY{p}{)}
         
         \PY{n}{fig}\PY{o}{.}\PY{n}{savefig}\PY{p}{(}\PY{l+s+s1}{\PYZsq{}}\PY{l+s+s1}{fig\PYZus{}boxcar.png}\PY{l+s+s1}{\PYZsq{}}\PY{p}{)}
\end{Verbatim}


    \begin{center}
    \adjustimage{max size={0.9\linewidth}{0.9\paperheight}}{output_8_0.png}
    \end{center}
    { \hspace*{\fill} \\}
    
    \hypertarget{histogramas}{%
\subsection{Histogramas}\label{histogramas}}

Finalmente un ejemplo para mostrar una distribución de una serie de
datos, que muestra por ejemplo si los datos tienen una distribución
normal (gaussiana) y de otro tipo. La figura muestra un histograma de
los datos y una cruva de una distribución normal.

El comando en este caso es \texttt{hist(X,bins=25,\ normed=True)} que
muestra el histograma de los datos dividida en 25 bins para el rango de
los datos. Se han removido los valores de los ejes.

    \begin{Verbatim}[commandchars=\\\{\}]
{\color{incolor}In [{\color{incolor}15}]:} \PY{c+c1}{\PYZsh{} plot\PYZus{}histogram.py}
         \PY{c+c1}{\PYZsh{} A simple histogram examples}
         
         \PY{k+kn}{import} \PY{n+nn}{matplotlib}\PY{n+nn}{.}\PY{n+nn}{pyplot} \PY{k}{as} \PY{n+nn}{plt}
         \PY{k+kn}{import} \PY{n+nn}{numpy} \PY{k}{as} \PY{n+nn}{np}
         
         
         \PY{n}{rd} \PY{o}{=} \PY{n}{np}\PY{o}{.}\PY{n}{random}\PY{o}{.}\PY{n}{RandomState}\PY{p}{(}\PY{p}{)}
         \PY{n}{X} \PY{o}{=} \PY{n}{rd}\PY{o}{.}\PY{n}{randn}\PY{p}{(}\PY{l+m+mi}{10000}\PY{p}{)}
         
         \PY{n}{fig}\PY{p}{,} \PY{n}{ax} \PY{o}{=} \PY{n}{plt}\PY{o}{.}\PY{n}{subplots}\PY{p}{(}\PY{p}{)}
         \PY{n}{ax}\PY{o}{.}\PY{n}{hist}\PY{p}{(}\PY{n}{X}\PY{p}{,} \PY{n}{bins}\PY{o}{=}\PY{l+m+mi}{25}\PY{p}{,} \PY{n}{density}\PY{o}{=}\PY{k+kc}{True}\PY{p}{)}
         \PY{n}{x} \PY{o}{=} \PY{n}{np}\PY{o}{.}\PY{n}{linspace}\PY{p}{(}\PY{o}{\PYZhy{}}\PY{l+m+mi}{5}\PY{p}{,} \PY{l+m+mi}{5}\PY{p}{,} \PY{l+m+mi}{1000}\PY{p}{)}
         \PY{n}{ax}\PY{o}{.}\PY{n}{plot}\PY{p}{(}\PY{n}{x}\PY{p}{,} \PY{l+m+mi}{1} \PY{o}{/} \PY{n}{np}\PY{o}{.}\PY{n}{sqrt}\PY{p}{(}\PY{l+m+mi}{2}\PY{o}{*}\PY{n}{np}\PY{o}{.}\PY{n}{pi}\PY{p}{)} \PY{o}{*} \PY{n}{np}\PY{o}{.}\PY{n}{exp}\PY{p}{(}\PY{o}{\PYZhy{}}\PY{p}{(}\PY{n}{x}\PY{o}{*}\PY{o}{*}\PY{l+m+mi}{2}\PY{p}{)}\PY{o}{/}\PY{l+m+mi}{2}\PY{p}{)}\PY{p}{,} \PY{n}{linewidth}\PY{o}{=}\PY{l+m+mi}{4}\PY{p}{)}
         \PY{n}{ax}\PY{o}{.}\PY{n}{set\PYZus{}xticks}\PY{p}{(}\PY{p}{[}\PY{p}{]}\PY{p}{)}
         \PY{n}{ax}\PY{o}{.}\PY{n}{set\PYZus{}yticks}\PY{p}{(}\PY{p}{[}\PY{p}{]}\PY{p}{)}
         \PY{n}{plt}\PY{o}{.}\PY{n}{savefig}\PY{p}{(}\PY{l+s+s1}{\PYZsq{}}\PY{l+s+s1}{fig3.png}\PY{l+s+s1}{\PYZsq{}}\PY{p}{)}
         \PY{n}{plt}\PY{o}{.}\PY{n}{show}\PY{p}{(}\PY{p}{)} 
\end{Verbatim}


    \begin{center}
    \adjustimage{max size={0.9\linewidth}{0.9\paperheight}}{output_10_0.png}
    \end{center}
    { \hspace*{\fill} \\}
    
    \hypertarget{going-back-to-our-tp-data}{%
\section{Going back to our T/P data}\label{going-back-to-our-tp-data}}

For work in class

Take the metadata (\texttt{BASIN\_CHARACTERISTICS.CSV}) with the
location information of each site and plot a map with the station
locations.

Select one of the stations, and plot the temperature and precipitation
data in anopther figure.

    \begin{Verbatim}[commandchars=\\\{\}]
{\color{incolor}In [{\color{incolor}41}]:} \PY{c+c1}{\PYZsh{}}
         \PY{c+c1}{\PYZsh{} Reading CSV table with T and P data, and station information. }
         \PY{c+c1}{\PYZsh{}}
         
         \PY{k+kn}{import} \PY{n+nn}{numpy} \PY{k}{as} \PY{n+nn}{np}
         \PY{k+kn}{import} \PY{n+nn}{matplotlib}\PY{n+nn}{.}\PY{n+nn}{pyplot} \PY{k}{as} \PY{n+nn}{plt}
         \PY{k+kn}{import} \PY{n+nn}{cartopy}\PY{n+nn}{.}\PY{n+nn}{crs} \PY{k}{as} \PY{n+nn}{ccrs}
         
         \PY{c+c1}{\PYZsh{} Data file names}
         \PY{n}{Tfile}    \PY{o}{=} \PY{l+s+s1}{\PYZsq{}}\PY{l+s+s1}{MON\PYZus{}T\PYZus{}CRU\PYZus{}19012015.csv}\PY{l+s+s1}{\PYZsq{}}
         \PY{n}{Pfile}    \PY{o}{=} \PY{l+s+s1}{\PYZsq{}}\PY{l+s+s1}{MON\PYZus{}P\PYZus{}CRU\PYZus{}19012015.csv}\PY{l+s+s1}{\PYZsq{}}
         \PY{n}{sta\PYZus{}file} \PY{o}{=} \PY{l+s+s1}{\PYZsq{}}\PY{l+s+s1}{BASIN\PYZus{}CHARACTERISTICS.csv}\PY{l+s+s1}{\PYZsq{}}
         
         \PY{c+c1}{\PYZsh{} Read data files}
         \PY{n}{basin} \PY{o}{=} \PY{n}{np}\PY{o}{.}\PY{n}{loadtxt}\PY{p}{(}\PY{n}{sta\PYZus{}file}\PY{p}{,}\PY{n}{skiprows}\PY{o}{=}\PY{l+m+mi}{1}\PY{p}{,}\PY{n}{delimiter}\PY{o}{=}\PY{l+s+s1}{\PYZsq{}}\PY{l+s+s1}{,}\PY{l+s+s1}{\PYZsq{}}\PY{p}{)}
         \PY{n}{Temp}  \PY{o}{=} \PY{n}{np}\PY{o}{.}\PY{n}{loadtxt}\PY{p}{(}\PY{n}{Tfile}\PY{p}{,}\PY{n}{skiprows}\PY{o}{=}\PY{l+m+mi}{1}\PY{p}{,}\PY{n}{delimiter}\PY{o}{=}\PY{l+s+s1}{\PYZsq{}}\PY{l+s+s1}{,}\PY{l+s+s1}{\PYZsq{}}\PY{p}{)}
         \PY{n}{Prec}  \PY{o}{=} \PY{n}{np}\PY{o}{.}\PY{n}{loadtxt}\PY{p}{(}\PY{n}{Pfile}\PY{p}{,}\PY{n}{skiprows}\PY{o}{=}\PY{l+m+mi}{1}\PY{p}{,}\PY{n}{delimiter}\PY{o}{=}\PY{l+s+s1}{\PYZsq{}}\PY{l+s+s1}{,}\PY{l+s+s1}{\PYZsq{}}\PY{p}{)}
         
         \PY{c+c1}{\PYZsh{}}
         \PY{c+c1}{\PYZsh{} Organize data files}
         \PY{c+c1}{\PYZsh{}}
         
         \PY{c+c1}{\PYZsh{} Site column}
         \PY{n}{Tsite} \PY{o}{=} \PY{n}{Temp}\PY{p}{[}\PY{p}{:}\PY{p}{,}\PY{l+m+mi}{0}\PY{p}{]}
         \PY{n}{Psite} \PY{o}{=} \PY{n}{Prec}\PY{p}{[}\PY{p}{:}\PY{p}{,}\PY{l+m+mi}{0}\PY{p}{]}
         
         \PY{c+c1}{\PYZsh{} Data columns}
         \PY{n}{T} \PY{o}{=} \PY{n}{Temp}\PY{p}{[}\PY{p}{:}\PY{p}{,}\PY{l+m+mi}{1}\PY{p}{:}\PY{p}{]}
         \PY{n}{P} \PY{o}{=} \PY{n}{Prec}\PY{p}{[}\PY{p}{:}\PY{p}{,}\PY{l+m+mi}{1}\PY{p}{:}\PY{p}{]}
         
         \PY{n}{sta}  \PY{o}{=} \PY{n}{basin}\PY{p}{[}\PY{p}{:}\PY{p}{,}\PY{l+m+mi}{0}\PY{p}{]}
         \PY{n}{lon}  \PY{o}{=} \PY{n}{basin}\PY{p}{[}\PY{p}{:}\PY{p}{,}\PY{l+m+mi}{1}\PY{p}{]}
         \PY{n}{lat}  \PY{o}{=} \PY{n}{basin}\PY{p}{[}\PY{p}{:}\PY{p}{,}\PY{l+m+mi}{2}\PY{p}{]}
         \PY{n}{area} \PY{o}{=} \PY{n}{basin}\PY{p}{[}\PY{p}{:}\PY{p}{,}\PY{l+m+mi}{3}\PY{p}{]}
         \PY{n}{elev} \PY{o}{=} \PY{n}{basin}\PY{p}{[}\PY{p}{:}\PY{p}{,}\PY{l+m+mi}{4}\PY{p}{]}
         
         \PY{c+c1}{\PYZsh{} Confirm, stations are all the same order. }
         \PY{n+nb}{print}\PY{p}{(}\PY{n+nb}{max}\PY{p}{(}\PY{n+nb}{abs}\PY{p}{(}\PY{n}{Tsite}\PY{o}{\PYZhy{}}\PY{n}{sta}\PY{p}{)}\PY{p}{)}\PY{p}{)}
         \PY{n+nb}{print}\PY{p}{(}\PY{n+nb}{max}\PY{p}{(}\PY{n+nb}{abs}\PY{p}{(}\PY{n}{Psite}\PY{o}{\PYZhy{}}\PY{n}{sta}\PY{p}{)}\PY{p}{)}\PY{p}{)}
         
         \PY{c+c1}{\PYZsh{} Select station}
         \PY{n}{ista} \PY{o}{=} \PY{l+m+mi}{20}
         
         \PY{c+c1}{\PYZsh{} Figure with the map}
         
         \PY{n}{fig} \PY{o}{=} \PY{n}{plt}\PY{o}{.}\PY{n}{figure}\PY{p}{(}\PY{p}{)}
         \PY{n}{ax}  \PY{o}{=} \PY{n}{fig}\PY{o}{.}\PY{n}{add\PYZus{}subplot}\PY{p}{(}\PY{l+m+mi}{1}\PY{p}{,}\PY{l+m+mi}{1}\PY{p}{,}\PY{l+m+mi}{1}\PY{p}{,}\PY{n}{projection}\PY{o}{=}\PY{n}{ccrs}\PY{o}{.}\PY{n}{PlateCarree}\PY{p}{(}\PY{p}{)}\PY{p}{)}
         \PY{n}{ax}\PY{o}{.}\PY{n}{set\PYZus{}extent}\PY{p}{(}\PY{p}{[}\PY{o}{\PYZhy{}}\PY{l+m+mi}{180}\PY{p}{,} \PY{l+m+mi}{180}\PY{p}{,} \PY{o}{\PYZhy{}}\PY{l+m+mi}{90}\PY{p}{,} \PY{l+m+mi}{90}\PY{p}{]}\PY{p}{)}
         \PY{c+c1}{\PYZsh{}ax.stock\PYZus{}img()}
         \PY{n}{ax}\PY{o}{.}\PY{n}{coastlines}\PY{p}{(}\PY{p}{)}
         \PY{n}{plt}\PY{o}{.}\PY{n}{plot}\PY{p}{(}\PY{n}{lon}\PY{p}{,}\PY{n}{lat}\PY{p}{,}\PY{l+s+s1}{\PYZsq{}}\PY{l+s+s1}{k\PYZca{}}\PY{l+s+s1}{\PYZsq{}}\PY{p}{)}
         \PY{n}{plt}\PY{o}{.}\PY{n}{plot}\PY{p}{(}\PY{n}{lon}\PY{p}{[}\PY{n}{ista}\PY{p}{]}\PY{p}{,}\PY{n}{lat}\PY{p}{[}\PY{n}{ista}\PY{p}{]}\PY{p}{,}\PY{l+s+s1}{\PYZsq{}}\PY{l+s+s1}{r\PYZca{}}\PY{l+s+s1}{\PYZsq{}}\PY{p}{)}
         
         \PY{c+c1}{\PYZsh{} Figure with the time series}
         \PY{n}{fig}\PY{p}{,} \PY{n}{ax1} \PY{o}{=} \PY{n}{plt}\PY{o}{.}\PY{n}{subplots}\PY{p}{(}\PY{p}{)}
         
         \PY{n}{color} \PY{o}{=} \PY{l+s+s1}{\PYZsq{}}\PY{l+s+s1}{tab:red}\PY{l+s+s1}{\PYZsq{}}
         \PY{c+c1}{\PYZsh{}ax1.set\PYZus{}xlabel(\PYZsq{}time (s)\PYZsq{})}
         \PY{n}{ax1}\PY{o}{.}\PY{n}{set\PYZus{}ylabel}\PY{p}{(}\PY{l+s+s1}{\PYZsq{}}\PY{l+s+s1}{Temperature}\PY{l+s+s1}{\PYZsq{}}\PY{p}{,} \PY{n}{color}\PY{o}{=}\PY{n}{color}\PY{p}{)}
         \PY{n}{ax1}\PY{o}{.}\PY{n}{plot}\PY{p}{(}\PY{n}{T}\PY{p}{[}\PY{n}{ista}\PY{p}{,}\PY{p}{:}\PY{p}{]}\PY{p}{,} \PY{n}{color}\PY{o}{=}\PY{n}{color}\PY{p}{)}
         \PY{n}{ax1}\PY{o}{.}\PY{n}{tick\PYZus{}params}\PY{p}{(}\PY{n}{axis}\PY{o}{=}\PY{l+s+s1}{\PYZsq{}}\PY{l+s+s1}{y}\PY{l+s+s1}{\PYZsq{}}\PY{p}{,} \PY{n}{labelcolor}\PY{o}{=}\PY{n}{color}\PY{p}{)}
         
         \PY{n}{ax2} \PY{o}{=} \PY{n}{ax1}\PY{o}{.}\PY{n}{twinx}\PY{p}{(}\PY{p}{)}  \PY{c+c1}{\PYZsh{} instantiate a second axes that shares the same x\PYZhy{}axis}
         
         \PY{n}{color} \PY{o}{=} \PY{l+s+s1}{\PYZsq{}}\PY{l+s+s1}{tab:blue}\PY{l+s+s1}{\PYZsq{}}
         \PY{n}{ax2}\PY{o}{.}\PY{n}{set\PYZus{}ylabel}\PY{p}{(}\PY{l+s+s1}{\PYZsq{}}\PY{l+s+s1}{Precipitation}\PY{l+s+s1}{\PYZsq{}}\PY{p}{,} \PY{n}{color}\PY{o}{=}\PY{n}{color}\PY{p}{)}  \PY{c+c1}{\PYZsh{} we already handled the x\PYZhy{}label with ax1}
         \PY{n}{ax2}\PY{o}{.}\PY{n}{plot}\PY{p}{(}\PY{n}{P}\PY{p}{[}\PY{n}{ista}\PY{p}{,}\PY{p}{:}\PY{p}{]}\PY{p}{,} \PY{n}{color}\PY{o}{=}\PY{n}{color}\PY{p}{)}
         \PY{n}{ax2}\PY{o}{.}\PY{n}{tick\PYZus{}params}\PY{p}{(}\PY{n}{axis}\PY{o}{=}\PY{l+s+s1}{\PYZsq{}}\PY{l+s+s1}{y}\PY{l+s+s1}{\PYZsq{}}\PY{p}{,} \PY{n}{labelcolor}\PY{o}{=}\PY{n}{color}\PY{p}{)}
         
         \PY{n}{fig}\PY{o}{.}\PY{n}{tight\PYZus{}layout}\PY{p}{(}\PY{p}{)}  \PY{c+c1}{\PYZsh{} otherwise the right y\PYZhy{}label is slightly clipped}
         \PY{n}{plt}\PY{o}{.}\PY{n}{show}\PY{p}{(}\PY{p}{)}
         
         \PY{n}{ax1}\PY{o}{.}\PY{n}{set\PYZus{}xlim}\PY{p}{(}\PY{l+m+mi}{1000}\PY{p}{,} \PY{l+m+mi}{1200}\PY{p}{)}
         \PY{n}{plt}\PY{o}{.}\PY{n}{show}\PY{p}{(}\PY{p}{)}
         
         
         \PY{c+c1}{\PYZsh{} Figure with the time series}
         \PY{n}{fig}\PY{p}{,} \PY{n}{ax1} \PY{o}{=} \PY{n}{plt}\PY{o}{.}\PY{n}{subplots}\PY{p}{(}\PY{p}{)}
         
         \PY{n}{color} \PY{o}{=} \PY{l+s+s1}{\PYZsq{}}\PY{l+s+s1}{tab:red}\PY{l+s+s1}{\PYZsq{}}
         \PY{c+c1}{\PYZsh{}ax1.set\PYZus{}xlabel(\PYZsq{}time (s)\PYZsq{})}
         \PY{n}{ax1}\PY{o}{.}\PY{n}{set\PYZus{}ylabel}\PY{p}{(}\PY{l+s+s1}{\PYZsq{}}\PY{l+s+s1}{Temperature}\PY{l+s+s1}{\PYZsq{}}\PY{p}{,} \PY{n}{color}\PY{o}{=}\PY{n}{color}\PY{p}{)}
         \PY{n}{ax1}\PY{o}{.}\PY{n}{plot}\PY{p}{(}\PY{n}{T}\PY{p}{[}\PY{n}{ista}\PY{p}{,}\PY{p}{:}\PY{p}{]}\PY{p}{,} \PY{n}{color}\PY{o}{=}\PY{n}{color}\PY{p}{)}
         \PY{n}{ax1}\PY{o}{.}\PY{n}{tick\PYZus{}params}\PY{p}{(}\PY{n}{axis}\PY{o}{=}\PY{l+s+s1}{\PYZsq{}}\PY{l+s+s1}{y}\PY{l+s+s1}{\PYZsq{}}\PY{p}{,} \PY{n}{labelcolor}\PY{o}{=}\PY{n}{color}\PY{p}{)}
         
         \PY{n}{ax2} \PY{o}{=} \PY{n}{ax1}\PY{o}{.}\PY{n}{twinx}\PY{p}{(}\PY{p}{)}  \PY{c+c1}{\PYZsh{} instantiate a second axes that shares the same x\PYZhy{}axis}
         
         \PY{n}{color} \PY{o}{=} \PY{l+s+s1}{\PYZsq{}}\PY{l+s+s1}{tab:blue}\PY{l+s+s1}{\PYZsq{}}
         \PY{n}{ax2}\PY{o}{.}\PY{n}{set\PYZus{}ylabel}\PY{p}{(}\PY{l+s+s1}{\PYZsq{}}\PY{l+s+s1}{Precipitation}\PY{l+s+s1}{\PYZsq{}}\PY{p}{,} \PY{n}{color}\PY{o}{=}\PY{n}{color}\PY{p}{)}  \PY{c+c1}{\PYZsh{} we already handled the x\PYZhy{}label with ax1}
         \PY{n}{ax2}\PY{o}{.}\PY{n}{plot}\PY{p}{(}\PY{n}{P}\PY{p}{[}\PY{n}{ista}\PY{p}{,}\PY{p}{:}\PY{p}{]}\PY{p}{,} \PY{n}{color}\PY{o}{=}\PY{n}{color}\PY{p}{)}
         \PY{n}{ax2}\PY{o}{.}\PY{n}{tick\PYZus{}params}\PY{p}{(}\PY{n}{axis}\PY{o}{=}\PY{l+s+s1}{\PYZsq{}}\PY{l+s+s1}{y}\PY{l+s+s1}{\PYZsq{}}\PY{p}{,} \PY{n}{labelcolor}\PY{o}{=}\PY{n}{color}\PY{p}{)}
         
         \PY{n}{ax1}\PY{o}{.}\PY{n}{set\PYZus{}xlim}\PY{p}{(}\PY{l+m+mi}{1000}\PY{p}{,} \PY{l+m+mi}{1200}\PY{p}{)}
         
         \PY{n}{fig}\PY{o}{.}\PY{n}{tight\PYZus{}layout}\PY{p}{(}\PY{p}{)}  \PY{c+c1}{\PYZsh{} otherwise the right y\PYZhy{}label is slightly clipped}
         
         \PY{n}{plt}\PY{o}{.}\PY{n}{show}\PY{p}{(}\PY{p}{)}
\end{Verbatim}


    \begin{Verbatim}[commandchars=\\\{\}]
0.0
0.0

    \end{Verbatim}

    \begin{center}
    \adjustimage{max size={0.9\linewidth}{0.9\paperheight}}{output_12_1.png}
    \end{center}
    { \hspace*{\fill} \\}
    
    \begin{center}
    \adjustimage{max size={0.9\linewidth}{0.9\paperheight}}{output_12_2.png}
    \end{center}
    { \hspace*{\fill} \\}
    
    \begin{center}
    \adjustimage{max size={0.9\linewidth}{0.9\paperheight}}{output_12_3.png}
    \end{center}
    { \hspace*{\fill} \\}
    
    \hypertarget{gruxe1ficas-2d3d}{%
\section{Gráficas 2D/3D}\label{gruxe1ficas-2d3d}}

En Geociencias, muchas veces se toman datos en la superficie de la
Tierra y se quieren mostrar en mapa o en sección cruzada o en figura 3D.
Sin embargo, es importante tener en cuenta que las figuras en 3D muchas
veces no son muy útiles para mostrar los datos. En los siguientes
ejemplos muestro 3-4 formas de presentar unos datos z tomados en la
posición x y y donde z puede representar por ejemplo altura, alguna
anomalía geofísica, contenido de SiO2, etc.

La figura muestra una figura tipo scatter donde cada punto representa un
dato tomado en campo con tres valores \texttt{x\ y\ z}. La figura
muestra la posición \texttt{(x,y)} y en color el valor correspondiente a
\texttt{z}.

\hypertarget{nota}{%
\paragraph{NOTA}\label{nota}}

La figura muestra la suma de dos montículos gaussianos, uno con un pico
muy alto. Si la escala de colores de z es lineal, la figura muestra sólo
un pico. En cambio, si se usa una escala de colores logarítmica
(norm=cm.LogNorm()), si se puede observar mucho mejor la tendencia de
los dos montículos.

    \begin{Verbatim}[commandchars=\\\{\}]
{\color{incolor}In [{\color{incolor}41}]:} \PY{c+c1}{\PYZsh{} plot\PYZus{}scatter.py}
         
         \PY{k+kn}{import} \PY{n+nn}{matplotlib}\PY{n+nn}{.}\PY{n+nn}{colors} \PY{k}{as} \PY{n+nn}{cm}
         \PY{k+kn}{import} \PY{n+nn}{matplotlib}\PY{n+nn}{.}\PY{n+nn}{pyplot} \PY{k}{as} \PY{n+nn}{plt}
         \PY{k+kn}{import} \PY{n+nn}{numpy} \PY{k}{as} \PY{n+nn}{np}
         \PY{k+kn}{import} \PY{n+nn}{matplotlib}\PY{n+nn}{.}\PY{n+nn}{mlab} \PY{k}{as} \PY{n+nn}{ml}
         
         \PY{n}{N} \PY{o}{=} \PY{l+m+mi}{2000}
         \PY{n}{x} \PY{o}{=} \PY{p}{(}\PY{n}{np}\PY{o}{.}\PY{n}{random}\PY{o}{.}\PY{n}{rand}\PY{p}{(}\PY{n}{N}\PY{p}{)}\PY{o}{\PYZhy{}}\PY{l+m+mf}{0.5}\PY{p}{)}\PY{o}{*}\PY{l+m+mi}{6}
         \PY{n}{y} \PY{o}{=} \PY{p}{(}\PY{n}{np}\PY{o}{.}\PY{n}{random}\PY{o}{.}\PY{n}{rand}\PY{p}{(}\PY{n}{N}\PY{p}{)}\PY{o}{\PYZhy{}}\PY{l+m+mf}{0.5}\PY{p}{)}\PY{o}{*}\PY{l+m+mi}{4}
         
         \PY{n}{z1} \PY{o}{=} \PY{n}{np}\PY{o}{.}\PY{n}{random}\PY{o}{.}\PY{n}{multivariate\PYZus{}normal}\PY{p}{(}\PY{n}{mean} \PY{o}{=} \PY{p}{[}\PY{l+m+mi}{0}\PY{p}{,} \PY{l+m+mi}{0}\PY{p}{]}\PY{p}{,} \PY{n}{cov} \PY{o}{=} \PY{n}{np}\PY{o}{.}\PY{n}{array}\PY{p}{(}\PY{p}{[}\PY{p}{[}\PY{l+m+mf}{0.1}\PY{p}{,} \PY{l+m+mf}{0.2}\PY{p}{]}\PY{p}{,}\PY{p}{[}\PY{l+m+mf}{0.25}\PY{p}{,} \PY{l+m+mf}{0.50}\PY{p}{]}\PY{p}{]}\PY{p}{)}\PY{p}{,} \PY{n}{size} \PY{o}{=} \PY{l+m+mi}{100000}\PY{p}{)}
         
         \PY{n}{z} \PY{o}{=} \PY{p}{(}\PY{n}{ml}\PY{o}{.}\PY{n}{bivariate\PYZus{}normal}\PY{p}{(}\PY{n}{x}\PY{p}{,} \PY{n}{y}\PY{p}{,} \PY{l+m+mf}{0.1}\PY{p}{,} \PY{l+m+mf}{0.2}\PY{p}{,} \PY{l+m+mf}{1.0}\PY{p}{,} \PY{l+m+mf}{1.0}\PY{p}{)}
              \PY{o}{+} \PY{l+m+mf}{0.5} \PY{o}{*} \PY{n}{ml}\PY{o}{.}\PY{n}{bivariate\PYZus{}normal}\PY{p}{(}\PY{n}{x}\PY{p}{,} \PY{n}{y}\PY{p}{,} \PY{l+m+mf}{0.5}\PY{p}{,} \PY{l+m+mf}{0.5}\PY{p}{,} \PY{l+m+mf}{0.0}\PY{p}{,} \PY{l+m+mf}{0.0}\PY{p}{)}\PY{p}{)}
         
         \PY{n}{z} \PY{o}{=} \PY{n}{z}\PY{o}{+}\PY{l+m+mf}{1e\PYZhy{}5}
         
         \PY{n}{plt}\PY{o}{.}\PY{n}{subplot}\PY{p}{(}\PY{l+m+mi}{212}\PY{p}{)}
         \PY{n}{plt}\PY{o}{.}\PY{n}{scatter}\PY{p}{(}\PY{n}{x}\PY{p}{,} \PY{n}{y}\PY{p}{,} \PY{n}{c}\PY{o}{=}\PY{n}{z}\PY{p}{,} \PY{n}{norm}\PY{o}{=}\PY{n}{cm}\PY{o}{.}\PY{n}{LogNorm}\PY{p}{(}\PY{p}{)}\PY{p}{,} \PY{n}{cmap}\PY{o}{=}\PY{l+s+s1}{\PYZsq{}}\PY{l+s+s1}{winter}\PY{l+s+s1}{\PYZsq{}}\PY{p}{)}
         \PY{n}{plt}\PY{o}{.}\PY{n}{colorbar}\PY{p}{(}\PY{p}{)}
         \PY{n}{plt}\PY{o}{.}\PY{n}{title}\PY{p}{(}\PY{l+s+s1}{\PYZsq{}}\PY{l+s+s1}{Log color scale}\PY{l+s+s1}{\PYZsq{}}\PY{p}{)}
         
         \PY{n}{plt}\PY{o}{.}\PY{n}{subplot}\PY{p}{(}\PY{l+m+mi}{211}\PY{p}{)}
         \PY{n}{plt}\PY{o}{.}\PY{n}{scatter}\PY{p}{(}\PY{n}{x}\PY{p}{,} \PY{n}{y}\PY{p}{,} \PY{n}{c}\PY{o}{=}\PY{n}{z}\PY{p}{,} \PY{n}{cmap}\PY{o}{=}\PY{l+s+s1}{\PYZsq{}}\PY{l+s+s1}{winter}\PY{l+s+s1}{\PYZsq{}}\PY{p}{)}
         \PY{n}{plt}\PY{o}{.}\PY{n}{colorbar}\PY{p}{(}\PY{p}{)}
         \PY{n}{frame1} \PY{o}{=} \PY{n}{plt}\PY{o}{.}\PY{n}{gca}\PY{p}{(}\PY{p}{)}
         \PY{n}{frame1}\PY{o}{.}\PY{n}{axes}\PY{o}{.}\PY{n}{get\PYZus{}xaxis}\PY{p}{(}\PY{p}{)}\PY{o}{.}\PY{n}{set\PYZus{}ticks}\PY{p}{(}\PY{p}{[}\PY{p}{]}\PY{p}{)}
         \PY{n}{plt}\PY{o}{.}\PY{n}{title}\PY{p}{(}\PY{l+s+s1}{\PYZsq{}}\PY{l+s+s1}{Linear color scale}\PY{l+s+s1}{\PYZsq{}}\PY{p}{)}
         
         \PY{n}{plt}\PY{o}{.}\PY{n}{savefig}\PY{p}{(}\PY{l+s+s1}{\PYZsq{}}\PY{l+s+s1}{fig4.png}\PY{l+s+s1}{\PYZsq{}}\PY{p}{)}
         \PY{n}{plt}\PY{o}{.}\PY{n}{show}\PY{p}{(}\PY{p}{)}
\end{Verbatim}


    \begin{Verbatim}[commandchars=\\\{\}]
/Applications/anaconda3/lib/python3.6/site-packages/ipykernel\_launcher.py:12: MatplotlibDeprecationWarning: The bivariate\_normal function was deprecated in Matplotlib 2.2 and will be removed in 3.1.
  if sys.path[0] == '':
/Applications/anaconda3/lib/python3.6/site-packages/ipykernel\_launcher.py:13: MatplotlibDeprecationWarning: The bivariate\_normal function was deprecated in Matplotlib 2.2 and will be removed in 3.1.
  del sys.path[0]

    \end{Verbatim}

    \begin{center}
    \adjustimage{max size={0.9\linewidth}{0.9\paperheight}}{output_14_1.png}
    \end{center}
    { \hspace*{\fill} \\}
    
    El código \texttt{plot\_scatter.py} muestra el uso de
\texttt{plt.scatter}, que en realidad es una forma de mostrar unos datos
con tres dimensiones en plano 2D. Python tiene la opción para hacer
figuras tipo scatter en 3D también, aunque su uso no lo recomiendo,
porque no muestra los datos con facilidad.

Como último ejemplo, se muestra el uso de contornos o superficies en
2D/3D. Las figuras siguientes muestran los mismos datos de la figura de
scatter pero que fueron tomados en una grilla constante y no aleatoria.
Aunque esto no es lo que uno esperaría obtener en campo (es difícil
hacer una grilla perfecta), se requeriría utilizar códigos para
interpolación que se puede obtener en varios paquetes de Python (pero
que está más allá de lo que se busca en este capítulo).

En la primera figura se muestra los contornos para la variable
\texttt{z}, con curvas de nivel marcadas, o con curvas de nivel rellenas
con colores proporcionales a la altura. Note que en este caso se muestra
\texttt{z} en escala logarítmica.

    \begin{Verbatim}[commandchars=\\\{\}]
{\color{incolor}In [{\color{incolor}42}]:} \PY{c+c1}{\PYZsh{} plot\PYZus{}contour.py}
         
         \PY{k+kn}{from} \PY{n+nn}{mpl\PYZus{}toolkits}\PY{n+nn}{.}\PY{n+nn}{mplot3d} \PY{k}{import} \PY{n}{Axes3D}
         \PY{k+kn}{import} \PY{n+nn}{matplotlib}\PY{n+nn}{.}\PY{n+nn}{colors} \PY{k}{as} \PY{n+nn}{cm}
         \PY{k+kn}{import} \PY{n+nn}{matplotlib}\PY{n+nn}{.}\PY{n+nn}{pyplot} \PY{k}{as} \PY{n+nn}{plt}
         \PY{k+kn}{import} \PY{n+nn}{numpy} \PY{k}{as} \PY{n+nn}{np}
         \PY{k+kn}{import} \PY{n+nn}{matplotlib}\PY{n+nn}{.}\PY{n+nn}{mlab} \PY{k}{as} \PY{n+nn}{ml}
         
         \PY{n}{delta} \PY{o}{=} \PY{l+m+mf}{0.025}
         \PY{n}{x}     \PY{o}{=} \PY{n}{np}\PY{o}{.}\PY{n}{arange}\PY{p}{(}\PY{o}{\PYZhy{}}\PY{l+m+mf}{3.0}\PY{p}{,} \PY{l+m+mf}{3.0}\PY{p}{,}\PY{n}{delta}\PY{p}{)}
         \PY{n}{y}     \PY{o}{=} \PY{n}{np}\PY{o}{.}\PY{n}{arange}\PY{p}{(}\PY{o}{\PYZhy{}}\PY{l+m+mf}{2.0}\PY{p}{,} \PY{l+m+mf}{2.0}\PY{p}{,} \PY{n}{delta}\PY{p}{)}
         \PY{n}{X}\PY{p}{,} \PY{n}{Y}  \PY{o}{=} \PY{n}{np}\PY{o}{.}\PY{n}{meshgrid}\PY{p}{(}\PY{n}{x}\PY{p}{,}\PY{n}{y}\PY{p}{)}
         
         
         \PY{n}{z} \PY{o}{=} \PY{p}{(}\PY{n}{ml}\PY{o}{.}\PY{n}{bivariate\PYZus{}normal}\PY{p}{(}\PY{n}{X}\PY{p}{,} \PY{n}{Y}\PY{p}{,} \PY{l+m+mf}{0.1}\PY{p}{,} \PY{l+m+mf}{0.2}\PY{p}{,} \PY{l+m+mf}{1.0}\PY{p}{,} \PY{l+m+mf}{1.0}\PY{p}{)}
              \PY{o}{+} \PY{l+m+mf}{0.5} \PY{o}{*} \PY{n}{ml}\PY{o}{.}\PY{n}{bivariate\PYZus{}normal}\PY{p}{(}\PY{n}{X}\PY{p}{,} \PY{n}{Y}\PY{p}{,} \PY{l+m+mf}{0.5}\PY{p}{,} \PY{l+m+mf}{0.5}\PY{p}{,} \PY{l+m+mf}{0.0}\PY{p}{,} \PY{l+m+mf}{0.0}\PY{p}{)}\PY{p}{)}
         \PY{n}{z} \PY{o}{=} \PY{n}{z}\PY{o}{+}\PY{l+m+mf}{1e\PYZhy{}5}
         
         \PY{n}{plt}\PY{o}{.}\PY{n}{figure}\PY{p}{(}\PY{l+m+mi}{1}\PY{p}{)}
         \PY{n}{plt}\PY{o}{.}\PY{n}{subplot}\PY{p}{(}\PY{l+m+mi}{212}\PY{p}{)}
         \PY{n}{CS} \PY{o}{=} \PY{n}{plt}\PY{o}{.}\PY{n}{contourf}\PY{p}{(}\PY{n}{X}\PY{p}{,} \PY{n}{Y}\PY{p}{,} \PY{n}{z}\PY{p}{,}\PY{n}{norm}\PY{o}{=}\PY{n}{cm}\PY{o}{.}\PY{n}{LogNorm}\PY{p}{(}\PY{p}{)}\PY{p}{)}
         \PY{n}{plt}\PY{o}{.}\PY{n}{colorbar}\PY{p}{(}\PY{p}{)}
         \PY{n}{plt}\PY{o}{.}\PY{n}{title}\PY{p}{(}\PY{l+s+s1}{\PYZsq{}}\PY{l+s+s1}{Contour filled}\PY{l+s+s1}{\PYZsq{}}\PY{p}{)}
         
         \PY{n}{plt}\PY{o}{.}\PY{n}{subplot}\PY{p}{(}\PY{l+m+mi}{211}\PY{p}{)}
         \PY{n}{CS} \PY{o}{=} \PY{n}{plt}\PY{o}{.}\PY{n}{contour}\PY{p}{(}\PY{n}{X}\PY{p}{,} \PY{n}{Y}\PY{p}{,} \PY{n}{z}\PY{p}{,}\PY{n}{norm}\PY{o}{=}\PY{n}{cm}\PY{o}{.}\PY{n}{LogNorm}\PY{p}{(}\PY{p}{)}\PY{p}{,}\PY{n}{colors}\PY{o}{=}\PY{l+s+s1}{\PYZsq{}}\PY{l+s+s1}{k}\PY{l+s+s1}{\PYZsq{}}\PY{p}{)}
         \PY{n}{plt}\PY{o}{.}\PY{n}{clabel}\PY{p}{(}\PY{n}{CS}\PY{p}{,}\PY{n}{inline}\PY{o}{=}\PY{l+m+mi}{1}\PY{p}{,}\PY{n}{fontsize}\PY{o}{=}\PY{l+m+mi}{10}\PY{p}{,}\PY{n}{fmt}\PY{o}{=}\PY{l+s+s1}{\PYZsq{}}\PY{l+s+si}{\PYZpc{}4.1e}\PY{l+s+s1}{\PYZsq{}}\PY{p}{)}
         \PY{n}{frame1} \PY{o}{=} \PY{n}{plt}\PY{o}{.}\PY{n}{gca}\PY{p}{(}\PY{p}{)}
         \PY{n}{frame1}\PY{o}{.}\PY{n}{axes}\PY{o}{.}\PY{n}{get\PYZus{}xaxis}\PY{p}{(}\PY{p}{)}\PY{o}{.}\PY{n}{set\PYZus{}ticks}\PY{p}{(}\PY{p}{[}\PY{p}{]}\PY{p}{)}
         \PY{n}{plt}\PY{o}{.}\PY{n}{colorbar}\PY{p}{(}\PY{p}{)}
         \PY{n}{plt}\PY{o}{.}\PY{n}{title}\PY{p}{(}\PY{l+s+s1}{\PYZsq{}}\PY{l+s+s1}{Log Contour plot}\PY{l+s+s1}{\PYZsq{}}\PY{p}{)}
         \PY{n}{plt}\PY{o}{.}\PY{n}{savefig}\PY{p}{(}\PY{l+s+s1}{\PYZsq{}}\PY{l+s+s1}{fig5.png}\PY{l+s+s1}{\PYZsq{}}\PY{p}{)}
         \PY{n}{plt}\PY{o}{.}\PY{n}{show}\PY{p}{(}\PY{p}{)}
         
         \PY{n}{fig}  \PY{o}{=} \PY{n}{plt}\PY{o}{.}\PY{n}{figure}\PY{p}{(}\PY{p}{)}
         \PY{n}{ax}   \PY{o}{=} \PY{n}{fig}\PY{o}{.}\PY{n}{gca}\PY{p}{(}\PY{n}{projection}\PY{o}{=}\PY{l+s+s1}{\PYZsq{}}\PY{l+s+s1}{3d}\PY{l+s+s1}{\PYZsq{}}\PY{p}{)}
         \PY{n}{surf} \PY{o}{=} \PY{n}{ax}\PY{o}{.}\PY{n}{plot\PYZus{}surface}\PY{p}{(}\PY{n}{X}\PY{p}{,} \PY{n}{Y}\PY{p}{,} \PY{n}{np}\PY{o}{.}\PY{n}{log10}\PY{p}{(}\PY{n}{z}\PY{p}{)}\PY{p}{,}\PY{n}{cmap}\PY{o}{=}\PY{l+s+s1}{\PYZsq{}}\PY{l+s+s1}{terrain}\PY{l+s+s1}{\PYZsq{}}\PY{p}{)}
         \PY{n}{plt}\PY{o}{.}\PY{n}{title}\PY{p}{(}\PY{l+s+s1}{\PYZsq{}}\PY{l+s+s1}{Surface log10(Z)}\PY{l+s+s1}{\PYZsq{}}\PY{p}{)}
         
         \PY{n}{plt}\PY{o}{.}\PY{n}{savefig}\PY{p}{(}\PY{l+s+s1}{\PYZsq{}}\PY{l+s+s1}{fig6.png}\PY{l+s+s1}{\PYZsq{}}\PY{p}{)}
         \PY{n}{plt}\PY{o}{.}\PY{n}{show}\PY{p}{(}\PY{p}{)}
\end{Verbatim}


    \begin{Verbatim}[commandchars=\\\{\}]
/Applications/anaconda3/lib/python3.6/site-packages/ipykernel\_launcher.py:15: MatplotlibDeprecationWarning: The bivariate\_normal function was deprecated in Matplotlib 2.2 and will be removed in 3.1.
  from ipykernel import kernelapp as app
/Applications/anaconda3/lib/python3.6/site-packages/ipykernel\_launcher.py:16: MatplotlibDeprecationWarning: The bivariate\_normal function was deprecated in Matplotlib 2.2 and will be removed in 3.1.
  app.launch\_new\_instance()

    \end{Verbatim}

    \begin{center}
    \adjustimage{max size={0.9\linewidth}{0.9\paperheight}}{output_16_1.png}
    \end{center}
    { \hspace*{\fill} \\}
    
    \begin{center}
    \adjustimage{max size={0.9\linewidth}{0.9\paperheight}}{output_16_2.png}
    \end{center}
    { \hspace*{\fill} \\}
    
    \hypertarget{going-back-to-our-tp-data-2}{%
\section{Going back to our T/P data
2}\label{going-back-to-our-tp-data-2}}

For work in class

With the data already read, plot the average Temperature in your
preferred color scale, for all sites, for the month of january, in map
view. Notice you can use \emph{scatter} plots or something similar. But
note you need to take the average temperature.

Make a second plot for the month of June instead.

Describe the patterns you see.

    \begin{Verbatim}[commandchars=\\\{\}]
{\color{incolor}In [{\color{incolor}65}]:} \PY{c+c1}{\PYZsh{}}
         \PY{c+c1}{\PYZsh{} Reading CSV table with T and P data, and station information. }
         \PY{c+c1}{\PYZsh{}}
         
         \PY{k+kn}{import} \PY{n+nn}{numpy} \PY{k}{as} \PY{n+nn}{np}
         \PY{k+kn}{import} \PY{n+nn}{matplotlib}\PY{n+nn}{.}\PY{n+nn}{pyplot} \PY{k}{as} \PY{n+nn}{plt}
         \PY{k+kn}{import} \PY{n+nn}{cartopy}\PY{n+nn}{.}\PY{n+nn}{crs} \PY{k}{as} \PY{n+nn}{ccrs}
         
         \PY{c+c1}{\PYZsh{} Data file names}
         \PY{n}{Tfile}    \PY{o}{=} \PY{l+s+s1}{\PYZsq{}}\PY{l+s+s1}{MON\PYZus{}T\PYZus{}CRU\PYZus{}19012015.csv}\PY{l+s+s1}{\PYZsq{}}
         \PY{n}{Pfile}    \PY{o}{=} \PY{l+s+s1}{\PYZsq{}}\PY{l+s+s1}{MON\PYZus{}P\PYZus{}CRU\PYZus{}19012015.csv}\PY{l+s+s1}{\PYZsq{}}
         \PY{n}{sta\PYZus{}file} \PY{o}{=} \PY{l+s+s1}{\PYZsq{}}\PY{l+s+s1}{BASIN\PYZus{}CHARACTERISTICS.csv}\PY{l+s+s1}{\PYZsq{}}
         
         \PY{c+c1}{\PYZsh{} Read data files}
         \PY{n}{basin} \PY{o}{=} \PY{n}{np}\PY{o}{.}\PY{n}{loadtxt}\PY{p}{(}\PY{n}{sta\PYZus{}file}\PY{p}{,}\PY{n}{skiprows}\PY{o}{=}\PY{l+m+mi}{1}\PY{p}{,}\PY{n}{delimiter}\PY{o}{=}\PY{l+s+s1}{\PYZsq{}}\PY{l+s+s1}{,}\PY{l+s+s1}{\PYZsq{}}\PY{p}{)}
         \PY{n}{Temp}  \PY{o}{=} \PY{n}{np}\PY{o}{.}\PY{n}{loadtxt}\PY{p}{(}\PY{n}{Tfile}\PY{p}{,}\PY{n}{skiprows}\PY{o}{=}\PY{l+m+mi}{1}\PY{p}{,}\PY{n}{delimiter}\PY{o}{=}\PY{l+s+s1}{\PYZsq{}}\PY{l+s+s1}{,}\PY{l+s+s1}{\PYZsq{}}\PY{p}{)}
         \PY{n}{Prec}  \PY{o}{=} \PY{n}{np}\PY{o}{.}\PY{n}{loadtxt}\PY{p}{(}\PY{n}{Pfile}\PY{p}{,}\PY{n}{skiprows}\PY{o}{=}\PY{l+m+mi}{1}\PY{p}{,}\PY{n}{delimiter}\PY{o}{=}\PY{l+s+s1}{\PYZsq{}}\PY{l+s+s1}{,}\PY{l+s+s1}{\PYZsq{}}\PY{p}{)}
         
         \PY{c+c1}{\PYZsh{}}
         \PY{c+c1}{\PYZsh{} Organize data files}
         \PY{c+c1}{\PYZsh{}}
         
         \PY{c+c1}{\PYZsh{} Site column}
         \PY{n}{Tsite} \PY{o}{=} \PY{n}{Temp}\PY{p}{[}\PY{p}{:}\PY{p}{,}\PY{l+m+mi}{0}\PY{p}{]}
         \PY{n}{Psite} \PY{o}{=} \PY{n}{Prec}\PY{p}{[}\PY{p}{:}\PY{p}{,}\PY{l+m+mi}{0}\PY{p}{]}
         
         \PY{c+c1}{\PYZsh{} Data columns}
         \PY{n}{T} \PY{o}{=} \PY{n}{Temp}\PY{p}{[}\PY{p}{:}\PY{p}{,}\PY{l+m+mi}{1}\PY{p}{:}\PY{p}{]}
         \PY{n}{P} \PY{o}{=} \PY{n}{Prec}\PY{p}{[}\PY{p}{:}\PY{p}{,}\PY{l+m+mi}{1}\PY{p}{:}\PY{p}{]}
         
         \PY{n}{sta}  \PY{o}{=} \PY{n}{basin}\PY{p}{[}\PY{p}{:}\PY{p}{,}\PY{l+m+mi}{0}\PY{p}{]}
         \PY{n}{lon}  \PY{o}{=} \PY{n}{basin}\PY{p}{[}\PY{p}{:}\PY{p}{,}\PY{l+m+mi}{1}\PY{p}{]}
         \PY{n}{lat}  \PY{o}{=} \PY{n}{basin}\PY{p}{[}\PY{p}{:}\PY{p}{,}\PY{l+m+mi}{2}\PY{p}{]}
         \PY{n}{area} \PY{o}{=} \PY{n}{basin}\PY{p}{[}\PY{p}{:}\PY{p}{,}\PY{l+m+mi}{3}\PY{p}{]}
         \PY{n}{elev} \PY{o}{=} \PY{n}{basin}\PY{p}{[}\PY{p}{:}\PY{p}{,}\PY{l+m+mi}{4}\PY{p}{]}
         
         \PY{n}{T01} \PY{o}{=} \PY{n}{np}\PY{o}{.}\PY{n}{mean}\PY{p}{(}\PY{n}{T}\PY{p}{[}\PY{p}{:}\PY{p}{,}\PY{l+m+mi}{0}\PY{p}{:}\PY{p}{:}\PY{l+m+mi}{12}\PY{p}{]}\PY{p}{,}\PY{l+m+mi}{1}\PY{p}{)}
         \PY{n}{T07} \PY{o}{=} \PY{n}{np}\PY{o}{.}\PY{n}{mean}\PY{p}{(}\PY{n}{T}\PY{p}{[}\PY{p}{:}\PY{p}{,}\PY{l+m+mi}{6}\PY{p}{:}\PY{p}{:}\PY{l+m+mi}{12}\PY{p}{]}\PY{p}{,}\PY{l+m+mi}{1}\PY{p}{)}
         \PY{n+nb}{print}\PY{p}{(}\PY{n}{T}\PY{o}{.}\PY{n}{shape}\PY{p}{,}\PY{n}{T01}\PY{o}{.}\PY{n}{shape}\PY{p}{,}\PY{n}{T01}\PY{p}{)}
         
         \PY{c+c1}{\PYZsh{} Figure with the map}
         
         \PY{n}{fig} \PY{o}{=} \PY{n}{plt}\PY{o}{.}\PY{n}{figure}\PY{p}{(}\PY{p}{)}
         \PY{n}{ax}  \PY{o}{=} \PY{n}{fig}\PY{o}{.}\PY{n}{add\PYZus{}subplot}\PY{p}{(}\PY{l+m+mi}{1}\PY{p}{,}\PY{l+m+mi}{1}\PY{p}{,}\PY{l+m+mi}{1}\PY{p}{,}\PY{n}{projection}\PY{o}{=}\PY{n}{ccrs}\PY{o}{.}\PY{n}{PlateCarree}\PY{p}{(}\PY{p}{)}\PY{p}{)}
         \PY{n}{ax}\PY{o}{.}\PY{n}{set\PYZus{}extent}\PY{p}{(}\PY{p}{[}\PY{o}{\PYZhy{}}\PY{l+m+mi}{180}\PY{p}{,} \PY{l+m+mi}{180}\PY{p}{,} \PY{o}{\PYZhy{}}\PY{l+m+mi}{90}\PY{p}{,} \PY{l+m+mi}{90}\PY{p}{]}\PY{p}{)}
         \PY{c+c1}{\PYZsh{}ax.stock\PYZus{}img()}
         \PY{n}{ax}\PY{o}{.}\PY{n}{coastlines}\PY{p}{(}\PY{p}{)}
         \PY{n}{plt}\PY{o}{.}\PY{n}{scatter}\PY{p}{(}\PY{n}{lon}\PY{p}{,}\PY{n}{lat}\PY{p}{,} \PY{n}{c}\PY{o}{=}\PY{n}{T01}\PY{p}{,} \PY{n}{cmap}\PY{o}{=}\PY{l+s+s1}{\PYZsq{}}\PY{l+s+s1}{rainbow}\PY{l+s+s1}{\PYZsq{}}\PY{p}{)}
         \PY{n}{plt}\PY{o}{.}\PY{n}{clim}\PY{p}{(}\PY{o}{\PYZhy{}}\PY{l+m+mi}{20}\PY{p}{,} \PY{l+m+mi}{30}\PY{p}{)}\PY{p}{;}
         \PY{n}{plt}\PY{o}{.}\PY{n}{colorbar}\PY{p}{(}\PY{n}{fraction}\PY{o}{=}\PY{l+m+mf}{0.04}\PY{p}{,}\PY{n}{pad}\PY{o}{=}\PY{l+m+mf}{0.04}\PY{p}{)}
         
         \PY{n}{fig} \PY{o}{=} \PY{n}{plt}\PY{o}{.}\PY{n}{figure}\PY{p}{(}\PY{p}{)}
         \PY{n}{ax}  \PY{o}{=} \PY{n}{fig}\PY{o}{.}\PY{n}{add\PYZus{}subplot}\PY{p}{(}\PY{l+m+mi}{1}\PY{p}{,}\PY{l+m+mi}{1}\PY{p}{,}\PY{l+m+mi}{1}\PY{p}{,}\PY{n}{projection}\PY{o}{=}\PY{n}{ccrs}\PY{o}{.}\PY{n}{PlateCarree}\PY{p}{(}\PY{p}{)}\PY{p}{)}
         \PY{n}{ax}\PY{o}{.}\PY{n}{set\PYZus{}extent}\PY{p}{(}\PY{p}{[}\PY{o}{\PYZhy{}}\PY{l+m+mi}{180}\PY{p}{,} \PY{l+m+mi}{180}\PY{p}{,} \PY{o}{\PYZhy{}}\PY{l+m+mi}{90}\PY{p}{,} \PY{l+m+mi}{90}\PY{p}{]}\PY{p}{)}
         \PY{c+c1}{\PYZsh{}ax.stock\PYZus{}img()}
         \PY{n}{ax}\PY{o}{.}\PY{n}{coastlines}\PY{p}{(}\PY{p}{)}
         \PY{n}{plt}\PY{o}{.}\PY{n}{scatter}\PY{p}{(}\PY{n}{lon}\PY{p}{,}\PY{n}{lat}\PY{p}{,} \PY{n}{c}\PY{o}{=}\PY{n}{T06}\PY{p}{,} \PY{n}{cmap}\PY{o}{=}\PY{l+s+s1}{\PYZsq{}}\PY{l+s+s1}{rainbow}\PY{l+s+s1}{\PYZsq{}}\PY{p}{)}
         \PY{n}{plt}\PY{o}{.}\PY{n}{clim}\PY{p}{(}\PY{o}{\PYZhy{}}\PY{l+m+mi}{20}\PY{p}{,} \PY{l+m+mi}{30}\PY{p}{)}\PY{p}{;}
         \PY{n}{plt}\PY{o}{.}\PY{n}{colorbar}\PY{p}{(}\PY{n}{fraction}\PY{o}{=}\PY{l+m+mf}{0.04}\PY{p}{,}\PY{n}{pad}\PY{o}{=}\PY{l+m+mf}{0.04}\PY{p}{)}
\end{Verbatim}


    \begin{Verbatim}[commandchars=\\\{\}]
(2673, 1380) (2673,) [25.70956522 25.5426087  23.93826087 {\ldots} -8.74033913 -4.9920687
 -7.54403478]

    \end{Verbatim}

\begin{Verbatim}[commandchars=\\\{\}]
{\color{outcolor}Out[{\color{outcolor}65}]:} <matplotlib.colorbar.Colorbar at 0xb1ff0e208>
\end{Verbatim}
            
    \begin{center}
    \adjustimage{max size={0.9\linewidth}{0.9\paperheight}}{output_18_2.png}
    \end{center}
    { \hspace*{\fill} \\}
    
    \begin{center}
    \adjustimage{max size={0.9\linewidth}{0.9\paperheight}}{output_18_3.png}
    \end{center}
    { \hspace*{\fill} \\}
    

    % Add a bibliography block to the postdoc
    
    
    
    \end{document}
